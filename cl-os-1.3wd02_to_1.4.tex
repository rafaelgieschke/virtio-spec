db68dc046b32 & 28 Sep 2023 & Cornelia Huck & { Revert "remove enumerate usage that makes the tool unhappy"
\newline


This reverts commit 3abace87db23ddceaf9688a405dd3fd540023977.
\newline

We can fix it properly instead.
\newline

Signed-off-by: Cornelia Huck <cohuck@redhat.com>\newline

 } \\
\hline
c7bef01491d0 & 28 Sep 2023 & Michael S. Tsirkin & { html: add missing enumitem package
\newline


makediffhtml.sh currently fails with:
\newline

! Missing number, treated as zero.
<to be read again>
                   \textbackslash c@*
l.25850 \textbackslash begin\{enumerate\}[label=\textbackslash alph*
                                      .]
?
! Emergency stop.
<to be read again>
                   \textbackslash c@*
l.25850 \textbackslash begin\{enumerate\}[label=\textbackslash alph*
                                      .]
\newline

Some web searches turned up suggestions to use enumitem and in fact,
virtio.tex already does this - but virtio-html.tex doesn't.
\newline

Adding \textbackslash usepackage\{enumitem\} in virtio-html.tex too fixes the issue.
\newline

Signed-off-by: Michael S. Tsirkin <mst@redhat.com>\newline
Signed-off-by: Cornelia Huck <cohuck@redhat.com>\newline

 } \\
\hline
3fdaa170307f & 30 Oct 2023 & jeshwank & { virtio-tee: Reserve device ID 46 for TEE device
\newline


In a virtual environment, an application running in guest VM may want
to delegate security sensitive tasks to a Trusted Application (TA)
running within a Trusted Execution Environment (TEE). A TEE is a trusted
OS running in some secure environment, for example, TrustZone on ARM
CPUs, or a separate secure co-processor etc.
\newline

A virtual TEE device emulates a TEE within a guest VM. Such a virtual
TEE device supports multiple operations such as:
\newline

VIRTIO_TEE_CMD_OPEN_DEVICE – Open a communication channel with virtio
                             TEE device.
VIRTIO_TEE_CMD_CLOSE_DEVICE – Close communication channel with virtio
                              TEE device.
VIRTIO_TEE_CMD_GET_VERSION – Get version of virtio TEE.
VIRTIO_TEE_CMD_OPEN_SESSION – Open a session to communicate with
                              trusted application running in TEE.
VIRTIO_TEE_CMD_CLOSE_SESSION – Close a session to end communication
                               with trusted application running in TEE.
VIRTIO_TEE_CMD_INVOKE_FUNC – Invoke a command or function in trusted
                             application running in TEE.
VIRTIO_TEE_CMD_CANCEL_REQ – Cancel an ongoing command within TEE.
VIRTIO_TEE_CMD_REGISTER_MEM - Register shared memory with TEE.
VIRTIO_TEE_CMD_UNREGISTER_MEM - Unregister shared memory from TEE.
\newline

We would like to reserve device ID 46 for Virtio-TEE device.
\newline

Fixes: \url{https://github.com/oasis-tcs/virtio-spec/issues/175}\newline

Signed-off-by: Jeshwanth Kumar <jeshwanthkumar.nk@amd.com>\newline
Reviewed-by: Rijo Thomas <Rijo-john.Thomas@amd.com>\newline
Reviewed-by: Parav Pandit <parav@nvidia.com>\newline
Acked-by: Sumit Garg <sumit.garg@linaro.org>\newline
Signed-off-by: Cornelia Huck <cohuck@redhat.com>\newline

 } \\
\hline
42f389989823 & 30 Oct 2023 & Xuan Zhuo & { virtio-net: support device stats
\newline


This patch allows the driver to obtain some statistics from the device.
\newline

In the device implementation, we can count a lot of such information,
which can be used for debugging and judging the running status of the
device. We hope to directly display it to the user through ethtool.
\newline

To get stats atomically, try to get stats for all/multiple queue pairs
in one command.
\newline

Fixes: \url{https://github.com/oasis-tcs/virtio-spec/issues/180}\newline

Signed-off-by: Xuan Zhuo <xuanzhuo@linux.alibaba.com>\newline
Suggested-by: Michael S. Tsirkin <mst@redhat.com>\newline
Reviewed-by: Parav Pandit <parav@nvidia.com>\newline
Signed-off-by: Cornelia Huck <cohuck@redhat.com>\newline

 } \\
\hline
925b42e3f72f & 30 Oct 2023 & Parav Pandit & { conformance: Add missing virtqueue reset conformance references
\newline


Add the missing references to the virtqueue reset related conformance
requirements.
\newline

Signed-off-by: Parav Pandit <parav@nvidia.com>\newline
Reviewed-by: Xuan Zhuo <xuanzhuo@linux.alibaba.com>\newline
[CH: pushed as an editorial change]
Signed-off-by: Cornelia Huck <cohuck@redhat.com>\newline

 } \\
\hline
68d5cc4fe408 & 30 Oct 2023 & Parav Pandit & { packed-ring: Change host,guest to device,driver
\newline


Rest of the packed ring description already uses the device
and driver terminology. Change the introductory line as well from
host and guest to device and driver respectively.
\newline

Signed-off-by: Parav Pandit <parav@nvidia.com>\newline
Acked-by: Michael S. Tsirkin <mst@redhat.com>\newline
[CH: pushed as an editorial update]
Signed-off-by: Cornelia Huck <cohuck@redhat.com>\newline

 } \\
\hline
c8249d73d2fd & 30 Oct 2023 & Cornelia Huck & { editorial: allow for longer device id table
\newline


Move to "longtable" to allow the table to span multiple pages (it
became too long to fit on one page with the latest addition.)
\newline

Signed-off-by: Cornelia Huck <cohuck@redhat.com>\newline

 } \\
\hline
eb16e3397177 & 03 Nov 2023 & Cornelia Huck & { editorial: various fixes for 1.3-csd01
\newline



- Set approval date to 06 October 2023. (applies to front page subtitle,
citation format, PDF page footers)
\newline

- Set filenames and URIs to show csd01 instead of wd01 (also PDF footers)
\newline

- Set "Previous stage" to "N/A" (don't list a different numbered Version)
\newline

- In "Related work", change (3x) text - "Latest version" to "Latest stage"
\newline

- In "Notices", set the copyright year to 2023.
\newline


- In the first line of Appendix D. Revision History, replace "the previous
version" with "Version 1.2"
\newline


- In Section 1.3, apply the current IETF-recommended text:
"The key words "MUST", "MUST NOT", "REQUIRED", "SHALL", "SHALL NOT",
"SHOULD", "SHOULD NOT", "RECOMMENDED", "NOT RECOMMENDED", "MAY", and
"OPTIONAL" in this document are to be interpreted as described in BCP 14
[[RFC2119](\#link)] and [[RFC8174](\#link)] when, and only when, they appear
in all capitals, as shown here."
\newline

- Also, add the RFC 8174 reference:
[RFC8174]
Leiba, B., "Ambiguity of Uppercase vs Lowercase in RFC 2119 Key Words", BCP
14, RFC 8174, DOI 10.17487/RFC8174, May 2017,
\url{https://www.rfc-editor.org/info/rfc8174.}
\newline

Reported-by: Paul Knight <paul.knight@oasis-open.org>\newline
Signed-off-by: Cornelia Huck <cohuck@redhat.com>\newline

 } \\
\hline
43a948ec6a92 & 03 Nov 2023 & Cornelia Huck & { editorial: update "Computer Language Definitions" URL
\newline


Split out from the other fixes for 1.3-csd01 so that we can fixup the
diff.
\newline


- In "Status", (fourth paragraph) change the hyperlink under (Computer
Language Definitions) to be "
\url{https://www.oasis-open.org/policies-guidelines/tc-process-2017-05-26/\#wpComponentsCompLang}
"
\newline

Reported-by: Paul Knight <paul.knight@oasis-open.org>\newline
Signed-off-by: Cornelia Huck <cohuck@redhat.com>\newline

 } \\
\hline
5fc35a7efb90 & 03 Nov 2023 & Cornelia Huck & { makediff: update list of cherry-picks
\newline


We don't need to apply the old commits anymore, but we have to apply
the URL update to work around a not-yet-debugged latexdiff problem.
\newline

Signed-off-by: Cornelia Huck <cohuck@redhat.com>\newline

 } \\
\hline
4cb03b12dc95 & 17 Nov 2023 & Parav Pandit & { description: Avoid splitting the word virtqueue
\newline


Don't split the word virtqueue.
\newline

Signed-off-by: Parav Pandit <parav@nvidia.com>\newline
[CH: applied as editorial]
Signed-off-by: Cornelia Huck <cohuck@redhat.com>\newline

 } \\
\hline
878a5e7287d7 & 13 Feb 2024 & Shujun Xue & { Define the DEVICE ID of Virtio Cpu balloon device as 47.
\newline


The Virtio CPU balloon device is a primitive device for managing guest
CPU capacity: the device asks for certain CPU cores to be online,
offline or throttled, and the driver performs the requested operation.
This allows the guest to adapt to changes in allowance of underlying
CPU capacity.
\newline

Fixes: \url{https://github.com/oasis-tcs/virtio-spec/issues/187}\newline

Signed-off-by: Shujun Xue <shujunxue@google.com>\newline
Signed-off-by: Cornelia Huck <cohuck@redhat.com>\newline

 } \\
\hline
07bb9f7f642d & 13 Feb 2024 & Harald Mommer & { virtio-can: Device specification.
\newline


virtio-can is a virtual CAN device. It provides a way to give access to
a CAN controller from a driver guest. The device is aimed to be used by
driver guests running a HLOS as well as by driver guests running a
typical RTOS as used in controller environments.
\newline

Fixes: \url{https://github.com/oasis-tcs/virtio-spec/issues/186}\newline

Signed-off-by: Harald Mommer <Harald.Mommer@opensynergy.com>\newline
Signed-off-by: Mikhail Golubev-Ciuchea <Mikhail.Golubev-Ciuchea@opensynergy.com>\newline
Signed-off-by: Cornelia Huck <cohuck@redhat.com>\newline

 } \\
\hline
193daee1d7a3 & 13 Feb 2024 & Heng Qi & { virtio-net: support the RSS context
\newline


Commit 84a1d9c48200 ("net: ethtool: extend RXNFC API to support RSS spreading of
filter matches") adds support for RSS context as a destination for receive flow filters
(see WIP work: \url{https://lists.oasis-open.org/archives/virtio-comment/202308/msg00194.html).}
\newline

An RSS context consists of configurable parameters specified by receive-side scaling.
\newline

Some use cases:
1. When users want some data flows to be steered to specific multiple rxqs, they can set
   receive flow filter rules for these data flows to an RSS context with desired rxqs.
2. Traffic isolation. Used when users want the traffic of certain applications to occupy
   several queues without being distubed.
\newline

How to set/configure an RSS context:
Assuming no RSS context has been created before.
1. ethtool -X eth0 context new start 5 equal 8
\newline

This command creates an RSS context with an id=1 for eth0, and fills in the indirection
table with rxq indexes 5-8 circularly. The hash key and hash types reuse the default
RSS configuration.
\newline

Then, we can use 'ethtool -x eth0 context 1' to query the above configuration.
\newline

2. ethtool -X eth0 context new start 6 equal 7 \textbackslash 
   hkey 8f:bf:dd:11:23:58:d2:8a:00:31:d0:32:a3:b5:1f:\textbackslash 
   1f:e4:d1:fe:47:7f:64:42:fd:d0:61:16:b8:b0:f9:71:e8:2d:36:7f:18:dd:4d:c8:f3
\newline

This command creates an RSS context with an id=2 for eth0, and fills in the indirection
table with rxq indexes 6-7 circularly. The hash key is 8f:bf:dd:11:23:58:d2:8a:00:31:d0\textbackslash 
:32:a3:b5:1f:1f:e4:d1:fe:47:7f:64:42:fd:d0:61:16:b8:b0:f9:71:e8:2d:36:7f:18:dd:4d:c8:f3.
Hash types reuse the default RSS configuration.
\newline

3. ethtool -N eth0 rx-flow-hash tcp4 sdfn context 1
\newline

This command specifies the hash types for the RSS context whose id=1 on eth0.
Now this RSS context only has the hash key to reuse the default RSS configuration.
\newline

4. ethtool -N eth0 flow-type udp4 src-ip 1.1.1.1 context 1
\newline

This command configures a receive flow filter rule for eth0, and the data flow matching
this rule will continue to select the final rxq according to the RSS context configuration
with id=1.
\newline

Fixes: \url{https://github.com/oasis-tcs/virtio-spec/issues/178}\newline

Signed-off-by: Heng Qi <hengqi@linux.alibaba.com>\newline
Signed-off-by: Parav Pandit <parav@nvidia.com>\newline
Acked-by: Satananda Burla <sburla@marvell.com>\newline
Acked-by: Xuan Zhuo <xuanzhuo@linux.alibaba.com>\newline
Signed-off-by: Cornelia Huck <cohuck@redhat.com>\newline

 } \\
\hline
bb11bf9b25cc & 13 Feb 2024 & Haixu Cui & { content: Rename SPI master to SPI controller
\newline


SPI master is an outdated term and should use SPI controller.
\newline

Signed-off-by: Haixu Cui <quic_haixcui@quicinc.com>\newline
Reviewed-by: Viresh Kumar <viresh.kumar@linaro.org>\newline
Signed-off-by: Cornelia Huck <cohuck@redhat.com>\newline

 } \\
\hline
a402acce814f & 13 Feb 2024 & Haixu Cui & { virtio-spi: add the device specification
\newline


The Virtio SPI (Serial Peripheral Interface) device is a virtual
SPI controller that allows the driver to operate and use the SPI
controller under the control of the host.
\newline

This patch adds the specification for virtio-spi.
\newline

Fixes: \url{https://github.com/oasis-tcs/virtio-spec/issues/189}\newline

Signed-off-by: Haixu Cui <quic_haixcui@quicinc.com>\newline
Reviewed-by: Viresh Kumar <viresh.kumar@linaro.org>\newline
Signed-off-by: Cornelia Huck <cohuck@redhat.com>\newline

 } \\
\hline
37c6a406678a & 16 Feb 2024 & Pape, Andreas (ADITG/ESS3) & { sound: add sampling rates 12000Hz and 24000Hz
\newline


24kHz is used for 'super wideband' voice transmission 12kHz is added 'for completeness'
\newline

Fixes: \url{https://github.com/oasis-tcs/virtio-spec/issues/184}\newline

Signed-off-by: Andreas Pape <apape@de.adit-jv.com>\newline
Reviewed-by: Anton Yakovlev <anton.yakovlev@opensynergy.com>\newline
Signed-off-by: Cornelia Huck <cohuck@redhat.com>\newline

 } \\
\hline
71fe8e907c9d & 21 May 2024 & Michael S. Tsirkin & { README.md: update mailing list info
\newline


As approved by the TC, we are moving to a less formal
way of discussing the specification, on the mailing lists
provided by the Linux Foundation:
\newline

\url{https://groups.oasis-open.org/higherlogic/ws/public/ballot?id=3820}
\newline

Update README.md, CONTRIBUTING.md and newdevice.tex accordingly.
\newline

Use this opportunity to explain when and how to use each
mailing list.
\newline

Oh yes, and device numbers are reserved through virtio-comment
not through virtio-dev. Correct that.
\newline

Message-Id: <\nolinkurl{8f5db33c96d685fcebca3579b05d09b64dd720d9.1715766697.git.mst@redhat.com}>\newline
Signed-off-by: Michael S. Tsirkin <mst@redhat.com>\newline

 } \\
\hline
6678bbf0ac15 & 11 Jul 2024 & Parav Pandit & { Add link for the feature bits section
\newline


Device common feature bits overview in the basic facilities and
their actual description are apart by 24 chapters.
\newline

Help reader to directly reach out to feature bits definitions from
the basic chapter.
\newline

MST: merged as a trivial editorial change\newline

Signed-off-by: Parav Pandit <parav@nvidia.com>\newline
Message-Id: <\nolinkurl{20240612073522.2571082-1-parav@nvidia.com}>\newline
Signed-off-by: Michael S. Tsirkin <mst@redhat.com>\newline

 } \\
\hline
550b76062270 & 11 Jul 2024 & Michael S. Tsirkin & { can: drop a broken conformance link
\newline


CAN device has a single device conformance section, drop
a link to a non-existent section, fixing a Latex error.
\newline

Fixes: 07bb9f7 ("virtio-can: Device specification.")\newline
Message-ID: <87a6c3c4463e9f304cdd5dd5fac2194a95a8bcd2.1720724591.git.mst@redhat.com>\newline
Signed-off-by: Michael S. Tsirkin <mst@redhat.com>\newline

 } \\
\hline
d7c486b49283 & 11 Jul 2024 & Michael S. Tsirkin & { introduction: fix label
\newline


label to rfc8174 is incorrect, leading to undefined reference
and multiple defined reference latex errors.
\newline

Fix it up.
\newline

Message-Id: <\nolinkurl{52b6fc16793e3b24284aefafda94322c739b9f7f.1720725332.git.mst@redhat.com}>\newline
Fixes: eb16e33 ("editorial: various fixes for 1.3-csd01")\newline
Signed-off-by: Michael S. Tsirkin <mst@redhat.com>\newline

 } \\
\hline
7e454b64747d & 11 Jul 2024 & Michael S. Tsirkin & { makediff: cherry pick label change
\newline


label changes tend to trip up makediff - this one
causes an undefined reference from deleted section.
To fix, cherry-pick it.
\newline

Message-Id: <\nolinkurl{216774e3fa811e0a6e994a3815f0804ddfc0bb03.1720731399.git.mst@redhat.com}>\newline
Signed-off-by: Michael S. Tsirkin <mst@redhat.com>\newline

 } \\
\hline
b495841a8e80 & 11 Jul 2024 & Michael S. Tsirkin & { transport-mmio: fix up makediff from 1.2
\newline


we fixed a typo in label name: in the diff old links become
undefined. Add the old label back for now, we can drop it
down the road.
\newline

Message-Id: <\nolinkurl{fc98549666e0fe74e890a7a62ca71248e709712b.1720731329.git.mst@redhat.com}>\newline
Fixes: ca97719 ("transport-mmio: Fix spellings and white spaces")\newline
Signed-off-by: Michael S. Tsirkin <mst@redhat.com>\newline

 } \\
\hline
75086cb72af6 & 11 Jul 2024 & Michael S. Tsirkin & { Merge tag 'v1.3-wd02' into virtio-1.4
\newline


Merge fixes from master branch (1.3) up to tag v1.3-wd02
commit b495841 ("transport-mmio: fix up makediff from 1.2")
\newline

Signed-off-by: Michael S. Tsirkin <mst@redhat.com>\newline

 } \\
\hline
61b65cbade3e & 11 Jul 2024 & Cornelia Huck & { editorial: allow for longer device id table: makediff 1.3
\newline


Move to "longtable" to allow the table to span multiple pages (it
became too long to fit on one page with the latest addition.)
\newline

Signed-off-by: Cornelia Huck <cohuck@redhat.com>\newline
(cherry picked from commit c8249d73d2fdbbfd38e8bf45c8492057bc2485e9)
\newline

 } \\
\hline
6fa0087aa8d3 & 11 Jul 2024 & Michael S. Tsirkin & { Merge tag 'v1.3-wd02-makediff' into virtio-1.4
\newline


this is just so makediff can cherry pick a workaround
for latexdiff.
\newline

Signed-off-by: Michael S. Tsirkin <mst@redhat.com>\newline

 } \\
\hline
79b04be2e022 & 11 Jul 2024 & Michael S. Tsirkin & { makediff: update base version to 1.3
\newline


might cause minor conflicts if we merge 1.3 branch
again, but seems cleaner than seeing full diff here.
\newline

Signed-off-by: Michael S. Tsirkin <mst@redhat.com>\newline

 } \\
\hline
18e47c01207b & 11 Jul 2024 & Parav Pandit & { virtio-blk: Fix data type of num_queues field
\newline


Correct the endianness of num_queues field to be little endian.
\newline

Suggested-by: Max Gurtovoy <mgurtovoy@nvidia.com>\newline
Reviewed-by: Max Gurtovoy <mgurtovoy@nvidia.com>\newline
Signed-off-by: Parav Pandit <parav@nvidia.com>\newline
Reviewed-by: Stefano Garzarella <sgarzare@redhat.com>\newline
Reviewed-by: Stefan Hajnoczi <stefanha@redhat.com>\newline
Message-Id: <\nolinkurl{20240521122119.2004071-1-parav@nvidia.com}>\newline
Signed-off-by: Michael S. Tsirkin <mst@redhat.com>\newline

 } \\
\hline
008068153853 & 11 Jul 2024 & Parav Pandit & { virtio-net: Fix receive buffer size calculation text
\newline


Receive buffer size calculation is based on the following
negotiated features.
\newline

The text has wrong calculation for IPv6 and also it has missed
VIRTIO_NET_F_HASH_REPORT.
\newline

The problem of igorance of VIRTIO_NET_F_HASH_REPORT is reported
in [1], however fix for ipv6 payload length must also be
considered.
\newline

Since for the both the fixes touching same requirements, a
new issue is created as [2].
\newline

This patch brings following fixes.
\newline

1. Fix annotating struct virtio_net_hdr as field
2. Fix receive buffer calculation for guest GSO cases to consider
   ipv6 payload length
3. small grammar corrections for article
4. reword the requirement to consider the virtio_ndr_hdr which is
   depends on the negotiated feature, hence first clarify the
   struct virtio_net_hdr size
\newline

[1] \url{https://github.com/oasis-tcs/virtio-spec/issues/170}
[2] \url{https://github.com/oasis-tcs/virtio-spec/issues/183}
\newline

Fixes: \url{https://github.com/oasis-tcs/virtio-spec/issues/170}\newline
Fixes: \url{https://github.com/oasis-tcs/virtio-spec/issues/183}\newline
Reviewed-by: Xuan Zhuo <xuanzhuo@linux.alibaba.com>\newline
Signed-off-by: Parav Pandit <parav@nvidia.com>\newline
Message-Id: <\nolinkurl{20240606102014.2103986-2-parav@nvidia.com}>\newline
Signed-off-by: Michael S. Tsirkin <mst@redhat.com>\newline

 } \\
\hline
bf0fdf8ba828 & 11 Jul 2024 & Parav Pandit & { virtio-net: Clarify the size of the struct virtio_net_hdr for tx
\newline


The feature VIRTIO_NET_F_HASH_REPORT only applies to the receive side.
However, when VIRTIO_NET_F_HASH_REPORT feature was introduced, it was
not clarified that the size of the struct virtio_net_hdr on the packet
transmission also uses higher size when VIRTIO_NET_F_HASH_REPORT is
negotiated.
\newline

Explicitly clarify this.
\newline

Fixes: \url{https://github.com/oasis-tcs/virtio-spec/issues/183}\newline
Reviewed-by: Xuan Zhuo <xuanzhuo@linux.alibaba.com>\newline
Signed-off-by: Parav Pandit <parav@nvidia.com>\newline
Message-Id: <\nolinkurl{20240606102014.2103986-3-parav@nvidia.com}>\newline
Signed-off-by: Michael S. Tsirkin <mst@redhat.com>\newline

 } \\
\hline
6abd42fd2398 & 11 Jul 2024 & Parav Pandit & { virtio-net: Annotate virtio_net_hdr as field
\newline


At several places struct virtio_net_hdr missed out the field
annotation. Add it.
\newline

Reviewed-by: Xuan Zhuo <xuanzhuo@linux.alibaba.com>\newline
Signed-off-by: Parav Pandit <parav@nvidia.com>\newline
Message-Id: <\nolinkurl{20240606102014.2103986-4-parav@nvidia.com}>\newline
Signed-off-by: Michael S. Tsirkin <mst@redhat.com>\newline

 } \\
\hline
fdd39cfa7d76 & 11 Jul 2024 & Parav Pandit & { admin: Introduce self group
\newline


Define self group to control the self device itself.
\newline

Subsequent patches introduces the concept of device capabilities
and device resources which utilizes the self group to access
capabilities and uses device resources to refer to the device itself.
\newline

Fixes: \url{https://github.com/oasis-tcs/virtio-spec/issues/179}\newline
Signed-off-by: Parav Pandit <parav@nvidia.com>\newline
Signed-off-by: Michael S. Tsirkin <mst@redhat.com>\newline
Acked-by: Satananda Burla <sburla@marvell.com>\newline
Message-Id: <\nolinkurl{20240604132903.2093195-2-parav@nvidia.com}>\newline
Signed-off-by: Michael S. Tsirkin <mst@redhat.com>\newline

 } \\
\hline
bfe175d9e018 & 11 Jul 2024 & Parav Pandit & { admin: Use already defined names for the legacy commands
\newline


Instead of description, use the existing name for defining the legacy
commands. While at it, prefer the shorter label names which
are already unique and refer them as hyperreference in the table
for quick naviation.
\newline

This is editorial change to align to subsequent patches.
\newline

Fixes: \url{https://github.com/oasis-tcs/virtio-spec/issues/179}\newline
Signed-off-by: Parav Pandit <parav@nvidia.com>\newline
Acked-by: Satananda Burla <sburla@marvell.com>\newline
Message-Id: <\nolinkurl{20240604132903.2093195-3-parav@nvidia.com}>\newline
Signed-off-by: Michael S. Tsirkin <mst@redhat.com>\newline

 } \\
\hline
88a24d28db5b & 11 Jul 2024 & Parav Pandit & { admin: Add theory of operation for capability admin commands
\newline


Device capability indicates the supported functionality and
resources of the device to the driver.
\newline

Driver capability indicates the supported functionality
and resources which driver will be using. Driver capability is
subset of the device capability.
\newline

Add theory of operation describing it.
\newline

Fixes: \url{https://github.com/oasis-tcs/virtio-spec/issues/179}\newline
Signed-off-by: Parav Pandit <parav@nvidia.com>\newline
Signed-off-by: Michael S. Tsirkin <mst@redhat.com>\newline
Acked-by: Satananda Burla <sburla@marvell.com>\newline
Message-Id: <\nolinkurl{20240604132903.2093195-4-parav@nvidia.com}>\newline
Signed-off-by: Michael S. Tsirkin <mst@redhat.com>\newline

 } \\
\hline
72a91e6d6365 & 11 Jul 2024 & Parav Pandit & { admin: Prepare table for multipage listing
\newline


If the table spans across two pages, it is not readable.
Make use of xltabular package that supports table spanning
across multiple pages.
\newline

Signed-off-by: Parav Pandit <parav@nvidia.com>\newline
Acked-by: Satananda Burla <sburla@marvell.com>\newline
Message-Id: <\nolinkurl{20240604132903.2093195-5-parav@nvidia.com}>\newline
Signed-off-by: Michael S. Tsirkin <mst@redhat.com>\newline

 } \\
\hline
d60a39072890 & 11 Jul 2024 & Parav Pandit & { admin: Add capability admin commands
\newline


Add three capabilities related administration commands.
First to get the device capability.
Second to set the driver capability.
Third for the driver to discover which capabilities can be accessed.
\newline

Even though the current series restricts device capability reading
to the self group type, same structure and command format etc will
be reusable in future to read the capability from the owner device
and also to set the device capability via owner device using
new DEV_CAP_SET command.
\newline

Resource objects are introduced in subsequent patch which utilizes
the capability, however some description around resource object
limit is covered in this patch to keep things simple.
\newline

Fixes: \url{https://github.com/oasis-tcs/virtio-spec/issues/179}\newline
Signed-off-by: Parav Pandit <parav@nvidia.com>\newline
Signed-off-by: Michael S. Tsirkin <mst@redhat.com>\newline
Acked-by: Satananda Burla <sburla@marvell.com>\newline
Message-Id: <\nolinkurl{20240604132903.2093195-6-parav@nvidia.com}>\newline
Signed-off-by: Michael S. Tsirkin <mst@redhat.com>\newline

 } \\
\hline
738f0a213307 & 11 Jul 2024 & Parav Pandit & { admin: Add theory of operation for device resource objects
\newline


The driver controls the device by means of device resource objects.
These operations include create, query, modify, and destroy the device
resource objects.
\newline

Fixes: \url{https://github.com/oasis-tcs/virtio-spec/issues/179}\newline
Signed-off-by: Parav Pandit <parav@nvidia.com>\newline
Signed-off-by: Michael S. Tsirkin <mst@redhat.com>\newline
Acked-by: Satananda Burla <sburla@marvell.com>\newline
Message-Id: <\nolinkurl{20240604132903.2093195-7-parav@nvidia.com}>\newline
Signed-off-by: Michael S. Tsirkin <mst@redhat.com>\newline

 } \\
\hline
5040f36bd8c9 & 11 Jul 2024 & Parav Pandit & { admin: Add device resource objects admin commands
\newline


Add generic administration commands create, modify, query and destroy
device resource object.
\newline

Each resource object defines resource specific attributes for the commands.
Once the resource object is created by the driver, it can be query/modify or
destroyed by the driver.
\newline

Each resource object is identified using a unique resource id per
resource type.
\newline

Fixes: \url{https://github.com/oasis-tcs/virtio-spec/issues/179}\newline
Signed-off-by: Parav Pandit <parav@nvidia.com>\newline
Signed-off-by: Michael S. Tsirkin <mst@redhat.com>\newline
Acked-by: Satananda Burla <sburla@marvell.com>\newline
Message-Id: <\nolinkurl{20240604132903.2093195-8-parav@nvidia.com}>\newline
Signed-off-by: Michael S. Tsirkin <mst@redhat.com>\newline

 } \\
\hline
7b3bdd63d821 & 11 Jul 2024 & Parav Pandit & { virtio-net: Add theory of operation for flow filter
\newline


Currently packet allow/drop interface has following limitations.
\newline

1. Driver can either select which MAC and VLANs to consider
for allowing/dropping packets, here, the driver has a
limitation that driver needs to supply full mac
table or full vlan table for each type. Driver cannot add or
delete an individual entry.
\newline

2. Driver cannot select mac+vlan combination for which
to allow/drop packet.
\newline

3. Driver cannot not set other commonly used packet match fields
such as IP header fields, TCP, UDP, SCP header fields.
\newline

4. Driver cannot steer specific packets based on the match
fields to specific receiveq.
\newline

5. Driver do not have multiple or dedicated virtqueues to
   perform flow filter requests in accelerated manner in
   the device.
\newline

Flow filter as a generic framework overcome above limitations.
\newline

As starting point it is useful to support at least two use cases.
a. ARFS
b. ethtool ntuple steering
\newline

In future it can be further extended for usecases such as
switching device, connection tracking or may be more.
\newline

The flow filter has following properties.
\newline

1. It is an extendible object that driver can create, destroy.
2. It belongs to a flow filter group.
3. Each flow filter rule is identified using a unique id,
   has priority, match key, destination(rq) and action(allow/drop).
4. Flow filter key also refers to the mask to learn which fields
   of the packets to match.
\newline

This patch adds theory of operation for flow filter functionality.
\newline

Fixes: \url{https://github.com/oasis-tcs/virtio-spec/issues/179}\newline
Signed-off-by: Parav Pandit <parav@nvidia.com>\newline
Signed-off-by: Heng Qi <hengqi@linux.alibaba.com>\newline
Signed-off-by: Michael S. Tsirkin <mst@redhat.com>\newline
Acked-by: Satananda Burla <sburla@marvell.com>\newline
Message-Id: <\nolinkurl{20240604132903.2093195-9-parav@nvidia.com}>\newline
Signed-off-by: Michael S. Tsirkin <mst@redhat.com>\newline

 } \\
\hline
899bb0ca24d8 & 11 Jul 2024 & Parav Pandit & { virtio-net: Add flow filter capability
\newline


Add first flow filter capability that indicates device resource
limits such as number of flow filter rules, groups and selectors.
\newline

add second capability that indicates supported selectors defining
which packet headers and their fields supported.
\newline

[Parav Pandit: resolve conflict: editorial: in spec links]
Fixes: \url{https://github.com/oasis-tcs/virtio-spec/issues/179}\newline
Signed-off-by: Parav Pandit <parav@nvidia.com>\newline
Signed-off-by: Heng Qi <hengqi@linux.alibaba.com>\newline
Signed-off-by: Michael S. Tsirkin <mst@redhat.com>\newline
Acked-by: Satananda Burla <sburla@marvell.com>\newline
Message-Id: <\nolinkurl{20240604132903.2093195-10-parav@nvidia.com}>\newline
Signed-off-by: Michael S. Tsirkin <mst@redhat.com>\newline

 } \\
\hline
224e98b0f53c & 11 Jul 2024 & Parav Pandit & { virtio-net: Add flow filter group, classifier and rule resource objects
\newline


Flow filter rules depend on the flow filter group resource object.
The device can have one or more flow filter groups. Each flow filter
group has its own order. The group order defines the packet
processing order in the flow filter domain.
\newline

Define flow filter classifier object which consists of one
or multiple types of packet header fields to consider
for matching. A mask can match on partial field or whole
field if the device supports partial masking for a given
mask type.
\newline

Define flow filter rule resource which consists of
match key, reference to group and mask objects and
an action.
\newline

Currently it covers the most common filter types and value
of Ethernet header, IPV4 and IPV6 headers and TCP and UDP headers.
\newline

Fixes: \url{https://github.com/oasis-tcs/virtio-spec/issues/179}\newline
Signed-off-by: Parav Pandit <parav@nvidia.com>\newline
Signed-off-by: Heng Qi <hengqi@linux.alibaba.com>\newline
Signed-off-by: Michael S. Tsirkin <mst@redhat.com>\newline
Acked-by: Satananda Burla <sburla@marvell.com>\newline
Message-Id: <\nolinkurl{20240604132903.2093195-11-parav@nvidia.com}>\newline
Signed-off-by: Michael S. Tsirkin <mst@redhat.com>\newline

 } \\
\hline
3b7b7371dbed & 11 Jul 2024 & Parav Pandit & { virtio-net: Add flow filter device and driver requirements
\newline


Add device and driver flow filter requirements.
\newline

Fixes: \url{https://github.com/oasis-tcs/virtio-spec/issues/179}\newline
Signed-off-by: Parav Pandit <parav@nvidia.com>\newline
Signed-off-by: Heng Qi <hengqi@linux.alibaba.com>\newline
Acked-by: Satananda Burla <sburla@marvell.com>\newline
Message-Id: <\nolinkurl{20240604132903.2093195-12-parav@nvidia.com}>\newline
Signed-off-by: Michael S. Tsirkin <mst@redhat.com>\newline

 } \\
\hline
9b3129fe7236 & 12 Jul 2024 & Martin Kröning & { virtio_pci_cap64: specify offset_hi, length_hi endianness
\newline


While the capability introduction says "This virtio structure capability uses little-endian format,"
it might be preferrable to be explicit about the endianness of offset_hi and length_hi.
\newline

Signed-off-by: Martin Kröning <martin.kroening@eonerc.rwth-aachen.de>\newline
Reviewed-by: Parav Pandit <parav@nvidia.com>\newline
Fixes: \url{https://github.com/oasis-tcs/virtio-spec/issues/196}\newline
Message-Id: <\nolinkurl{DF89BB0F-6BA4-4DAB-AEC3-03AAF858EC8E@eonerc.rwth-aachen.de}>\newline
Signed-off-by: Michael S. Tsirkin <mst@redhat.com>\newline

 } \\
\hline
67141d1cb0fe & 12 Jul 2024 & Parav Pandit & { admin: Add theory of operation for device parts
\newline


A device can get and set the state of the device which is
organized as multiple device parts. This can be useful
in cases of VM snapshot, VM migration, debug or some other use
cases.
\newline

Add the theory of operations on how to get/set device parts
and how it affects when the device is stopped or resumed.
\newline

Fixes: \url{https://github.com/oasis-tcs/virtio-spec/issues/176}\newline
Signed-off-by: Michael S. Tsirkin <mst@redhat.com>\newline
Signed-off-by: Parav Pandit <parav@nvidia.com>\newline
Message-Id: <\nolinkurl{20240601145042.2074739-2-parav@nvidia.com}>\newline

 } \\
\hline
a1281addd9b2 & 12 Jul 2024 & Parav Pandit & { admin: Extend resource objects for sr-iov group type
\newline


occasionally the group owner device accesses the device parts
of the member device. Extend the usage of resource objects
for member device parts access.
\newline

Fixes: \url{https://github.com/oasis-tcs/virtio-spec/issues/176}\newline
Signed-off-by: Parav Pandit <parav@nvidia.com>\newline
Message-Id: <\nolinkurl{20240601145042.2074739-3-parav@nvidia.com}>\newline

 } \\
\hline
aa4f6f069ab3 & 12 Jul 2024 & Parav Pandit & { admin: Add admin commands for device parts
\newline


Add administration commands to handle device parts such as
get/set device parts, get metadata of the device parts,
\newline

Fixes: \url{https://github.com/oasis-tcs/virtio-spec/issues/176}\newline
Signed-off-by: Parav Pandit <parav@nvidia.com>\newline
Message-Id: <\nolinkurl{20240601145042.2074739-4-parav@nvidia.com}>\newline

 } \\
\hline
617aa2d62a88 & 12 Jul 2024 & Parav Pandit & { admin: Define common device parts
\newline


Define common device parts that represents the state of the device.
The driver can get and set these device parts using administration
commands.
\newline

[Parav Pandit: resolve conflict: editorial: empty blank line at end of file newdevice.tex]
Fixes: \url{https://github.com/oasis-tcs/virtio-spec/issues/176}\newline
Signed-off-by: Parav Pandit <parav@nvidia.com>\newline
Message-Id: <\nolinkurl{20240601145042.2074739-5-parav@nvidia.com}>\newline

 } \\
\hline
52d320c8b3c5 & 12 Jul 2024 & Parav Pandit & { admin: Add requirements of device parts commands
\newline


Add device and driver side requirements for the device parts
related commands.
\newline

Fixes: \url{https://github.com/oasis-tcs/virtio-spec/issues/176}\newline
Signed-off-by: Parav Pandit <parav@nvidia.com>\newline
Message-Id: <\nolinkurl{20240601145042.2074739-6-parav@nvidia.com}>\newline

 } \\
\hline
24e93e10fdd4 & 15 Jul 2024 & Parav Pandit & { editorial: replace hyperref with ref
\newline


Replace hyperreference with the name reference.
Fix the broken reference link for the DEVICE_STATUS part.
\newline

Fixes: aa4f6f069ab3 ("admin: Add admin commands for device parts")\newline
Fixes: 617aa2d62a88 ("admin: Define common device parts")\newline
Suggested-by: Michael S. Tsirkin <mst@redhat.com>\newline
Signed-off-by: Parav Pandit <parav@nvidia.com>\newline
Message-Id: <\nolinkurl{20240714144023.3291803-1-parav@vr-arch-host06.mtvr.labs.mlnx}>\newline
Signed-off-by: Michael S. Tsirkin <mst@redhat.com>\newline

 } \\
\hline
737935487b9e & 15 Jul 2024 & Michael S. Tsirkin & { makediff: look in subjects only
\newline


We currently use rev-list with -F to find commits.
This helps if commit subject has any special characters
but means that pattern can apply anywhere in the line.
In particular, a common Fixes: (hash) "subject" pattern
we started using implies that fixups are picked up
instead of the commit.
\newline

For now, just switch to using regexp (dropping -F)
and add escape caret in front to only look for the string at
beginning of the line.
\newline

We'll worry about more elaborate logic later, if it's ever
needed.
\newline

Message-Id: <\nolinkurl{1130700f4ae71a93d9887af35cf105731219f057.1721062136.git.mst@redhat.com}>\newline
Signed-off-by: Michael S. Tsirkin <mst@redhat.com>\newline

 } \\
\hline
98ad5fcc88ab & 15 Jul 2024 & Michael S. Tsirkin & { makediff: cherry pick table env change
\newline


latexdiff can not cope with environment name changes.
cherry pick to avoid confusing it.
\newline

Fixes: 72a91e6 ("admin: Prepare table for multipage listing")\newline
Message-Id: <\nolinkurl{c210d1bf4c64145ad3aae7155ef6050433f23346.1721062136.git.mst@redhat.com}>\newline
Signed-off-by: Michael S. Tsirkin <mst@redhat.com>\newline

 } \\
\hline
849b421941a4 & 15 Jul 2024 & Michael S. Tsirkin & { admin: switch to tabularx
\newline


xltabular seems to confuse htlatex. Switch to an older
tabularx which seems to be sufficient for our purposes.
\newline

Message-Id: <\nolinkurl{29ebb999f6d3808edbf32cd6e95b69fefc1a6e34.1721062136.git.mst@redhat.com}>\newline
Signed-off-by: Michael S. Tsirkin <mst@redhat.com>\newline

 } \\
\hline
8e81624195ba & 15 Jul 2024 & Michael S. Tsirkin & { admin-cmds-device-parts: switch to tabularx
\newline


xltabular seems to confuse htlatex. Switch to an older
tabularx which seems to be sufficient for our purposes.
\newline

Message-Id: <\nolinkurl{bd3cf4e620485271d777f9cfc51d1ae9ee1b8169.1721062136.git.mst@redhat.com}>\newline
Signed-off-by: Michael S. Tsirkin <mst@redhat.com>\newline

 } \\
\hline
886fc333de8b & 15 Jul 2024 & Michael S. Tsirkin & { device-parts: switch to tabularx
\newline


xltabular seems to confuse htlatex. Switch to an older
tabularx which seems to be sufficient for our purposes.
\newline

Message-Id: <\nolinkurl{00f72154c624fbbe8c3da9dbdb4e4369c36b7f10.1721062136.git.mst@redhat.com}>\newline
Signed-off-by: Michael S. Tsirkin <mst@redhat.com>\newline

 } \\
\hline
a5fb7b5edd8a & 15 Jul 2024 & Michael S. Tsirkin & { admin-cmds-resource-objects: switch to tabularx
\newline


xltabular seems to confuse htlatex. Switch to an older
tabularx which seems to be sufficient for our purposes.
\newline

Message-Id: <\nolinkurl{3f7be88a42af0a384621ef07b65c7a31755ef42e.1721062136.git.mst@redhat.com}>\newline
Signed-off-by: Michael S. Tsirkin <mst@redhat.com>\newline

 } \\
\hline
459f1aa9ef74 & 15 Jul 2024 & Michael S. Tsirkin & { virtio: replace xltabular with ltablex
\newline


now that we only use tabularx, switch to ltablex to
make tabularx tables paginate properly.
\newline

Message-Id: <\nolinkurl{15b11ce5ac9bbe1aaa7818403b999ff39c74a479.1721062136.git.mst@redhat.com}>\newline
Signed-off-by: Michael S. Tsirkin <mst@redhat.com>\newline

 } \\
\hline
e48de848f976 & 15 Jul 2024 & Michael S. Tsirkin & { makediff: cherry pick switch to tabularx
\newline


Message-Id: <\nolinkurl{88cdb7020f5b2771a6d201b1f65005da97e54f16.1721062136.git.mst@redhat.com}>\newline
Signed-off-by: Michael S. Tsirkin <mst@redhat.com>\newline

 } \\
\hline
6c2340547951 & 15 Jul 2024 & Michael S. Tsirkin & { admin: get rid of _ in labels
\newline


it's not the 1st time we find out underscores in labels confuse
latexdiff machinery when generating html.
Errors look like this:
\newline

! Missing \textbackslash endcsname inserted.
<to be read again>
                   \textbackslash unhbox
....
\newline

?
! Emergency stop.
<to be read again>
                   \textbackslash unhbox
\newline

To fix, convert underscores in labels to dashes.
\newline

Message-Id: <\nolinkurl{3e773fcfd4246e66e6f39d3be95d2c0335e34532.1721062136.git.mst@redhat.com}>\newline
Signed-off-by: Michael S. Tsirkin <mst@redhat.com>\newline

 } \\
\hline
2a5a957a2571 & 23 Oct 2024 & Viresh Kumar & { virtio-transport: Add a section to define mandatory transport requirements
\newline


The current Virtio documentation lacks a set of generic requirements
applicable to all transports. Defining these generic requirements could
be beneficial when integrating support for a new transport.
\newline

This section outlines the essential requirements that any transport
method must adhere to.
\newline

Signed-off-by: Viresh Kumar <viresh.kumar@linaro.org>\newline
Fixes: \url{https://github.com/oasis-tcs/virtio-spec/issues/201}\newline
Reviewed-by: Stefano Garzarella <sgarzare@redhat.com>\newline
Link: \url{https://lore.kernel.org/r/4c24ede14e2545b2e9c69e7d9b79dc15744fd965.1723647318.git.viresh.kumar@linaro.org}\newline

 } \\
\hline
1d388440700d & 23 Oct 2024 & Parav Pandit & { device-parts: editorial: Add missing struct keyword
\newline


'struct' keyword was missing in the definition.
Add it.
\newline

Branch: virtio-1.4\newline
Fixes: aa4f6f06 ("admin: Add admin commands for device parts")\newline
Signed-off-by: Parav Pandit <parav@nvidia.com>\newline
Link: \url{https://lore.kernel.org/r/20240916030835.68178-2-parav@nvidia.com}\newline

 } \\
\hline
9081da1f1e58 & 23 Oct 2024 & Parav Pandit & { device-parts: editorial: Fix metadata type name
\newline


The description is for the type VIRTIO_ADMIN_CMD_DEV_PARTS_METADATA_TYPE_LIST;
however, it referred to a non-existent definition. Correct it.
\newline

Branch: virtio-1.4\newline
Fixes: aa4f6f069ab3 ("admin: Add admin commands for device parts")\newline
Fixes: \url{https://github.com/oasis-tcs/virtio-spec/issues/205}\newline
Reviewed-by: Matias Ezequiel Vara Larsen <mvaralar@redhat.com>\newline
Signed-off-by: Parav Pandit <parav@nvidia.com>\newline
Link: \url{https://lore.kernel.org/r/20240916030835.68178-3-parav@nvidia.com}\newline

 } \\
\hline
b820c61e4840 & 23 Oct 2024 & Parav Pandit & { crypto: editorial: Fix spelling errors
\newline


Fix spelling errors.
\newline

Branch: virtio-1.4\newline
Fixes: a385dd3366a2 ("virtio-crypto: Add virtio crypto device specification")\newline
Fixes: \url{https://github.com/oasis-tcs/virtio-spec/issues/205}\newline
Reviewed-by: Matias Ezequiel Vara Larsen <mvaralar@redhat.com>\newline
Signed-off-by: Parav Pandit <parav@nvidia.com>\newline
Link: \url{https://lore.kernel.org/r/20240916030835.68178-4-parav@nvidia.com}\newline

 } \\
\hline
ec1845bd5a02 & 23 Oct 2024 & Parav Pandit & { gpu: editorial: Fix spelling errors
\newline


Fix spelling errors.
\newline

Branch: virtio-1.4\newline
Fixes: fed64230bf31 ("Add virtio gpu device specification.")\newline
Fixes: \url{https://github.com/oasis-tcs/virtio-spec/issues/205}\newline
Reviewed-by: Matias Ezequiel Vara Larsen <mvaralar@redhat.com>\newline
Signed-off-by: Parav Pandit <parav@nvidia.com>\newline
Link: \url{https://lore.kernel.org/r/20240916030835.68178-5-parav@nvidia.com}\newline

 } \\
\hline
9d60d8457df0 & 23 Oct 2024 & Parav Pandit & { common: editorial: Fix spelling errors
\newline


Fix spelling errors.
\newline

Branch: virtio-1.4\newline
Fixes: 5f1a8ac61c15 ("admin: introduce virtio admin virtqueues")\newline
Fixes: 68f66ff7a3d9 ("content: define what an exported object is")\newline
Fixes: ef16b644cc25 ("content.tex: spec text converted to latex")\newline
Fixes: \url{https://github.com/oasis-tcs/virtio-spec/issues/205}\newline
Reviewed-by: Matias Ezequiel Vara Larsen <mvaralar@redhat.com>\newline
Signed-off-by: Parav Pandit <parav@nvidia.com>\newline
Link: \url{https://lore.kernel.org/r/20240916030835.68178-6-parav@nvidia.com}\newline

 } \\
\hline
e98070a9573f & 03 Nov 2024 & Albert Esteve & { content: Reserve ID for media device
\newline


Reserve device ID 49 for media device.
Compatible with video encoder, video decoder,
and capture devices.
\newline

Signed-off-by: Albert Esteve <aesteve@redhat.com>\newline
Fixes: \url{https://github.com/oasis-tcs/virtio-spec/issues/202}\newline
Link: \url{https://lore.kernel.org/r/20240906070833.1949727-1-aesteve@redhat.com}\newline

 } \\
\hline
417fb28849c6 & 03 Nov 2024 & Parav Pandit & { virtio-gpu: Fix spelling error
\newline


Fix spelling error.
\newline

Fixes: fed64230bf31 ("Add virtio gpu device specification.")\newline
Fixes: \url{https://github.com/oasis-tcs/virtio-spec/issues/192}\newline
Signed-off-by: Parav Pandit <parav@nvidia.com>\newline
Reviewed-by: Matias Ezequiel Vara Larsen <mvaralar@redhat.com>\newline

 } \\
\hline
d17c70ab971c & 03 Nov 2024 & Parav Pandit & { Revert "virtio-gpu: Fix spelling error"
\newline


This reverts commit 417fb28849c6ecac3b18a35e8a154c03f8dba66f.
Reverted due to missing Link in commit log.
\newline

Signed-off-by: Parav Pandit <parav@nvidia.com>\newline

 } \\
\hline
4a5df407eca3 & 03 Nov 2024 & Parav Pandit & { virtio-gpu: Fix spelling error
\newline


Fix spelling error.
\newline

Fixes: fed64230bf31 ("Add virtio gpu device specification.")\newline
Fixes: \url{https://github.com/oasis-tcs/virtio-spec/issues/192}\newline
Signed-off-by: Parav Pandit <parav@nvidia.com>\newline
Reviewed-by: Matias Ezequiel Vara Larsen <mvaralar@redhat.com>\newline
Link: \url{https://lore.kernel.org/r/20241027044457.495730-1-parav@nvidia.com}\newline

 } \\
\hline
a1849e31ddeb & 25 Nov 2024 & Juraj Marcin & { virtio-mem: introduce VIRTIO_MEM_F_PERSISTENT_SUSPEND
\newline


Before, the behavior while suspending to a deep sleep state and waking
up was not specified. For example, in x86 QEMU VM all devices receive a
reset request during wake-up. This would lead to unplugging of all the
plugged memory blocks. Due to this, suspending is disallowed in the
Linux Kernel, when plugged memory is present and
VIRTIO_MEM_F_PERSISTENT_SUSPEND feature flag is not advertised by the
virtio-mem device.
\newline

This new flag should signal to the guest driver, that the device can
correctly suspend to a deep sleep state and then wake up without
disrupting the plugged memory blocks. This feature flag is already
supported in the Linux Kernel [1] and also supported in QEMU [2].
\newline

[1]: \url{https://lore.kernel.org/all/20240318120645.105664-1-david@redhat.com/}
[2]: \url{https://lore.kernel.org/all/20240904103722.946194-1-jmarcin@redhat.com/}
\newline

Reviewed-by: David Hildenbrand <david@redhat.com>\newline
Reviewed-by: Matias Ezequiel Vara Larsen <mvaralar@redhat.com>\newline
Signed-off-by: Juraj Marcin <jmarcin@redhat.com>\newline
Fixes: \url{https://github.com/oasis-tcs/virtio-spec/issues/207}\newline
Link: \url{https://lore.kernel.org/r/20241009090202.10544-2-jmarcin@redhat.com}\newline

 } \\
\hline
c8c01f247ef7 & 25 Nov 2024 & Parav Pandit & { device-parts: editorial: Replace duplicated part type
\newline


Device part type VIRTIO_DEV_PART_VQ_CFG was duplicated in
the description; it is supposed to be VIRTIO_DEV_PART_VQ_NOTIFY_CFG.
Fix it.
\newline

Fixes: \url{https://github.com/oasis-tcs/virtio-spec/issues/209}\newline
Fixes: 617aa2d62a88 ("Signed-off-by: Parav Pandit <parav@nvidia.com>")\newline
Reviewed-by: Matias Ezequiel Vara Larsen <mvaralar@redhat.com>\newline
Signed-off-by: Parav Pandit <parav@nvidia.com>\newline
Link: \url{https://lore.kernel.org/r/20241020114141.451478-2-parav@nvidia.com}\newline

 } \\
\hline
f21f83160cf1 & 25 Nov 2024 & Parav Pandit & { device-parts: Add device type specific raw selector
\newline


The subsequent patch defines the device-type-specific parts. For
these parts, the raw selector format is defined to ensure that
each device type can specify its format accurately.
\newline

Fixes: \url{https://github.com/oasis-tcs/virtio-spec/issues/209}\newline
Signed-off-by: Parav Pandit <parav@nvidia.com>\newline
Reviewed-by: Matias Ezequiel Vara Larsen <mvaralar@redhat.com>\newline
Link: \url{https://lore.kernel.org/r/20241020114141.451478-3-parav@nvidia.com}\newline
Note: Merged with tab replaced with white spaces.\newline

 } \\
\hline
b2990c8a6642 & 25 Nov 2024 & Parav Pandit & { virtio-net: Define cvq configuration related device parts
\newline


virtio net driver sends the control virtqueue commands for
device configuration. Such driver configuration is currently
not captured in the device parts.
\newline

This series adds several of such device parts which represents
the network device specific configuration.
\newline

It is done by utilizing the existing device parts structure.
A new generic selector format is added to enable device type
specific device parts.
\newline

This series also reuses the existing control virtqueue command
structures, fields, and values to define the network device parts.
\newline

Fixes: \url{https://github.com/oasis-tcs/virtio-spec/issues/209}\newline
Reviewed-by: Matias Ezequiel Vara Larsen <mvaralar@redhat.com>\newline
Signed-off-by: Parav Pandit <parav@nvidia.com>\newline
Link: \url{https://lore.kernel.org/r/20241020114141.451478-4-parav@nvidia.com}\newline

 } \\
\hline
8e443f483843 & 20 Jan 2025 & Paolo Abeni & { virtio-net: clarify NEEDS_CSUM semantic for GSO packats.
\newline


The current wording is a bit unclear hinting to possible additional
nested headers. For GSO packets virtio net (currently) supports
offload for the checksum of single transport header, explicitly state
that in both the driver and device sections.
\newline

Signed-off-by: Paolo Abeni <pabeni@redhat.com>\newline
Reviewed-by: Jason Wang <jasowang@redhat.com>\newline
Reviewed-by: Parav Pandit <parav@nvidia.com>\newline
Link: \url{https://lore.kernel.org/r/81034d9f4b12f7bd7aa6f7f5266cb6d551d0823c.1732699986.git.pabeni@redhat.com}\newline

 } \\
\hline
95c085b2f16d & 20 Jan 2025 & Paolo Abeni & { virtio-net: clarify DATA_VALID semantic for encap protos.
\newline


DATA_VALID allows offloading a single checksum level, leaving
unspecified which header checksum is offloaded when one or more
encapsulated protocols are present.
\newline

In such a case, the only option usable from the guest OS is
offloading the outermost checksum. That also matches the existing
implementation.
\newline

Explicitly state such the constraint, to remove any ambiguity and
make later changes more straightforward.
\newline

Signed-off-by: Paolo Abeni <pabeni@redhat.com>\newline
Reviewed-by: Jason Wang <jasowang@redhat.com>\newline
Reviewed-by: Parav Pandit <parav@nvidia.com>\newline
Link: \url{https://lore.kernel.org/r/ec6fdbfba2c4fc1969d83799836ebb694a01fb30.1732699986.git.pabeni@redhat.com}\newline

 } \\
\hline
8cd457d8aa82 & 20 Jan 2025 & Paolo Abeni & { virtio-net: define UDP tunnel segmentation offload feature
\newline


The VIRTIO_NET_HDR_GSO_UDP_TUNNEL_IPV\{4,6\} are two gso_type flags
allowing respectively GSO over UDPv4 tunnel and GSO over UDPv6 tunnel.
They can be negotiated on both the host and guest sides.
\newline

One constraint addressed here is that the virtio side (either device or
driver) receiving a UDP tunneled GSO packet must be able to reconstruct
completely the inner and outer headers offset - to allow for later GSO.
\newline

To accommodate such need, new fields are introduced in the virtio_net
header: outer_th_offset and inner_nh_offset.
They map directly to the corresponding header information. The inner
transport header is implied by the (inner) checksum offload.
\newline

Those fields are required because a virtio device H/W implementation
performing segmentation for UDP tunneled packet will need to touch
the outer transport protocol (for the UDP length filed), the
inner network protocol (for the total length field, in the IPv4
case).
\newline

Note that segmentation will additionally need to touch
the outer network protocol and the inner transport protocol. The first
is implied/easily found with trivial parsing, the latter is identified
by the existing csum_start field.
\newline

Note that there is no concept of UDP tunnel type negotiation (e.g.
vxlan, geneve, vxlan-gpe, etc.), as a virtio device H/W implementation
can perform segmentation for every possible UDP-tunnel given the
specified new fields.
In the reverse direction, if a virtio device H/W implementation receives
some traffic for an unknown or unsupported UDP tunnel, it will simply
not aggregate the wire packet in a GSO one.
\newline

Signed-off-by: Paolo Abeni <pabeni@redhat.com>\newline
Reviewed-by: Jason Wang <jasowang@redhat.com>\newline
Link: \url{https://lore.kernel.org/r/3e1c45071c8cf4efc7ca0ecc7b52a73c6bf983fe.1732699986.git.pabeni@redhat.com}\newline

 } \\
\hline
3fea589bd7c6 & 20 Jan 2025 & Paolo Abeni & { virtio-net: define UDP tunnel checksum offload feature
\newline


This complements the previous change, additionally
introducing the UDP tunnel checksum offload feature.
\newline

Differently from the plain checksum offload feature, this
depends on UDP tunnel segmentation being available, as outer checksum
computation for non GSO packets is cheap and H/W implementation of
such a feature is complex.
\newline

UDP tunnel checksum offload does not introduce additional fields,
instead it leverages the outer transport offset introduced by the
UDP tunnel segmentation feature to locate the outer checksum
inside the packet.
\newline

When UDP tunnel checksum offload is negotiated:
\newline

- the driver requests the outer UDP checksum offload setting the
VIRTIO_NET_HDR_F_UDP_TUNNEL_CSUM bit in the flag field. Such bit
is not allocated inside the gso_type field to prevent space
exhaustion there.
\newline

- in the device -> driver direction, the VIRTIO_NET_HDR_F_DATA_VALID
bit semantic is extended, covering both the inner and the outer
checksum validation.
\newline

Signed-off-by: Paolo Abeni <pabeni@redhat.com>\newline
Reviewed-by: Jason Wang <jasowang@redhat.com>\newline
Link: \url{https://lore.kernel.org/r/e87cf0c61f5ea3783cea0bcc4ea74f6e73f453d7.1732699986.git.pabeni@redhat.com}\newline

 } \\
\hline
14d3f88cfd77 & 03 Feb 2025 & Parav Pandit & { virtio-net: Fix to avoid using reserved feature bits
\newline


Listed patches in the fixes tag, incorrectly used the reserved feature bits.
Fix them to use the well defined device specific range.
\newline

Fixes: \url{https://github.com/oasis-tcs/virtio-spec/issues/212}\newline
Fixes: \url{https://github.com/oasis-tcs/virtio-spec/issues/213}\newline
Fixes: 8cd457d8aa82 ("virtio-net: define UDP tunnel segmentation offload feature")\newline
Fixes: 3fea589bd7c6 ("virtio-net: define UDP tunnel checksum offload feature")\newline
Signed-off-by: Parav Pandit <parav@nvidia.com>\newline
Reviewed-by: Cornelia Huck <cohuck@redhat.com>\newline
Acked-by: Michael S. Tsirkin <mst@redhat.com>\newline
Link: \url{https://lore.kernel.org/r/20250126062058.13695-1-parav@nvidia.com}\newline

 } \\
\hline
124fcd0e97f2 & 24 Feb 2025 & Steffen Trumtrar & { virtio-net: Fix receive buffer size typo
\newline


The commit 00806815385340dd411cc67df3f6837935bb5e26 introduced a slight
typo in the struct virtio_net_hdr size calculation depending on
VIRTIO_NET_F_HASH_REPORT negotiation.
\newline

Without VIRTIO_NET_F_HASH_REPORT the struct is smaller than with the
feature. This mix up only occurs in one instance; sizes are correct in
all other occurences.
\newline

Fix this typo.
\newline

Fixes: 008068153853 ("virtio-net: Fix receive buffer size calculation text")\newline
Signed-off-by: Steffen Trumtrar <s.trumtrar@pengutronix.de>\newline
Reviewed-by: Parav Pandit <parav@nvidia.com>\newline
Acked-by: Michael S. Tsirkin <mst@redhat.com>\newline
Fixes: \url{https://github.com/oasis-tcs/virtio-spec/issues/216}\newline
Link: \url{https://lore.kernel.org/r/20250207-v1-4-topic-virtio-net-receive-buffer-fix-v1-1-efcef167d6bc@pengutronix.de}\newline

 } \\
\hline
a9b942e043d2 & 10 Mar 2025 & Parav Pandit & { virtio-net: Rename selectors_limit to classifiers_limit
\newline


A classifier object consists of one or more selectors. The number of
selectors per classifier object is already annotated in the field
selectors_per_classifier_limit.
\newline

The field selectors_limit is intended to reflect the limit of
classifier objects. Hence, rename it to classifiers_limit.
\newline

Fixes: \url{https://github.com/oasis-tcs/virtio-spec/issues/215}\newline
Fixes: 3b7b7371dbed ("virtio-net: Add flow filter device and driver requirements")\newline
Fixes: 899bb0ca24d8 ("virtio-net: Add flow filter capability")\newline
Signed-off-by: Parav Pandit <parav@nvidia.com>\newline
Reviewed-by: Matias Ezequiel Vara Larsen <mvaralar@redhat.com>\newline
Link: \url{https://lore.kernel.org/r/20250215061158.21053-1-parav@nvidia.com}\newline

 } \\
\hline
4c73a709a17a & 10 Apr 2025 & Parav Pandit & { conformance: Add missing virtqueue reset conformance references
\newline


Add the missing references to the virtqueue reset related conformance
requirements.
\newline

(cherry picked from commit 925b42e3f72fdd113a5e4cc219b739c2c74dba23)
\newline

Signed-off-by: Parav Pandit <parav@nvidia.com>\newline
Reviewed-by: Xuan Zhuo <xuanzhuo@linux.alibaba.com>\newline
[CH: pushed as an editorial change]
Signed-off-by: Cornelia Huck <cohuck@redhat.com>\newline
Signed-off-by: Matias Ezequiel Vara Larsen <mvaralar@redhat.com>\newline
Link: \url{https://lore.kernel.org/r/20250401133543.801184-2-mvaralar@redhat.com}\newline

 } \\
\hline
82b431347a5d & 10 Apr 2025 & Parav Pandit & { packed-ring: Change host,guest to device,driver
\newline


Rest of the packed ring description already uses the device
and driver terminology. Change the introductory line as well from
host and guest to device and driver respectively.
\newline

(cherry picked from commit 68d5cc4fe4081a49235885005647b095b7965c0b)
\newline

Signed-off-by: Parav Pandit <parav@nvidia.com>\newline
Acked-by: Michael S. Tsirkin <mst@redhat.com>\newline
[CH: pushed as an editorial update]
Signed-off-by: Cornelia Huck <cohuck@redhat.com>\newline
Signed-off-by: Matias Ezequiel Vara Larsen <mvaralar@redhat.com>\newline
Link: \url{https://lore.kernel.org/r/20250401133543.801184-3-mvaralar@redhat.com}\newline

 } \\
\hline
32a7417c4e99 & 10 Apr 2025 & Parav Pandit & { description: Avoid splitting the word virtqueue
\newline


Don't split the word virtqueue.
\newline

(cherry picked from commit 4cb03b12dc951f0152cd2cd9c79b24492e174e43)
\newline

Signed-off-by: Parav Pandit <parav@nvidia.com>\newline
[CH: applied as editorial]
Signed-off-by: Cornelia Huck <cohuck@redhat.com>\newline
Signed-off-by: Matias Ezequiel Vara Larsen <mvaralar@redhat.com>\newline
Link: \url{https://lore.kernel.org/r/20250401133543.801184-4-mvaralar@redhat.com}\newline

 } \\
\hline
36e90609bf5c & 10 Apr 2025 & Haixu Cui & { content: Rename SPI master to SPI controller
\newline


SPI master is an outdated term and should use SPI controller.
\newline

(cherry picked from commit bb11bf9b25cc86c6ff02bf9f243da55b0d383a32)
\newline

Signed-off-by: Haixu Cui <quic_haixcui@quicinc.com>\newline
Reviewed-by: Viresh Kumar <viresh.kumar@linaro.org>\newline
Signed-off-by: Cornelia Huck <cohuck@redhat.com>\newline
Signed-off-by: Matias Ezequiel Vara Larsen <mvaralar@redhat.com>\newline
Link: \url{https://lore.kernel.org/r/20250401133543.801184-5-mvaralar@redhat.com}\newline

 } \\
\hline
8297084d5fcd & 10 Apr 2025 & Parav Pandit & { virtio-blk: Fix data type of num_queues field
\newline


Correct the endianness of num_queues field to be little endian.
\newline

(cherry picked from commit 18e47c01207b2983a7dfd1e2c7de1bdc408d391a)
\newline

Suggested-by: Max Gurtovoy <mgurtovoy@nvidia.com>\newline
Reviewed-by: Max Gurtovoy <mgurtovoy@nvidia.com>\newline
Signed-off-by: Parav Pandit <parav@nvidia.com>\newline
Reviewed-by: Stefano Garzarella <sgarzare@redhat.com>\newline
Reviewed-by: Stefan Hajnoczi <stefanha@redhat.com>\newline
Message-Id: <\nolinkurl{20240521122119.2004071-1-parav@nvidia.com}>\newline
Signed-off-by: Michael S. Tsirkin <mst@redhat.com>\newline
Signed-off-by: Matias Ezequiel Vara Larsen <mvaralar@redhat.com>\newline
Link: \url{https://lore.kernel.org/r/20250401133543.801184-6-mvaralar@redhat.com}\newline

 } \\
\hline
a22cd3e8f6f4 & 10 Apr 2025 & Parav Pandit & { virtio-net: Fix receive buffer size calculation text
\newline


Receive buffer size calculation is based on the following
negotiated features.
\newline

The text has wrong calculation for IPv6 and also it has missed
VIRTIO_NET_F_HASH_REPORT.
\newline

The problem of igorance of VIRTIO_NET_F_HASH_REPORT is reported
in [1], however fix for ipv6 payload length must also be
considered.
\newline

Since for the both the fixes touching same requirements, a
new issue is created as [2].
\newline

This patch brings following fixes.
\newline

1. Fix annotating struct virtio_net_hdr as field
2. Fix receive buffer calculation for guest GSO cases to consider
   ipv6 payload length
3. small grammar corrections for article
4. reword the requirement to consider the virtio_ndr_hdr which is
   depends on the negotiated feature, hence first clarify the
   struct virtio_net_hdr size
\newline

[1] \url{https://github.com/oasis-tcs/virtio-spec/issues/170}
[2] \url{https://github.com/oasis-tcs/virtio-spec/issues/183}
\newline

(cherry picked from commit 00806815385340dd411cc67df3f6837935bb5e26)
\newline

Fixes: \url{https://github.com/oasis-tcs/virtio-spec/issues/170}\newline
Fixes: \url{https://github.com/oasis-tcs/virtio-spec/issues/183}\newline
Reviewed-by: Xuan Zhuo <xuanzhuo@linux.alibaba.com>\newline
Signed-off-by: Parav Pandit <parav@nvidia.com>\newline
Message-Id: <\nolinkurl{20240606102014.2103986-2-parav@nvidia.com}>\newline
Signed-off-by: Michael S. Tsirkin <mst@redhat.com>\newline
Signed-off-by: Matias Ezequiel Vara Larsen <mvaralar@redhat.com>\newline
Link: \url{https://lore.kernel.org/r/20250401133543.801184-7-mvaralar@redhat.com}\newline

 } \\
\hline
25d81013b392 & 10 Apr 2025 & Parav Pandit & { virtio-net: Clarify the size of the struct virtio_net_hdr for tx
\newline


The feature VIRTIO_NET_F_HASH_REPORT only applies to the receive side.
However, when VIRTIO_NET_F_HASH_REPORT feature was introduced, it was
not clarified that the size of the struct virtio_net_hdr on the packet
transmission also uses higher size when VIRTIO_NET_F_HASH_REPORT is
negotiated.
\newline

Explicitly clarify this.
\newline

(cherry picked from commit bf0fdf8ba828b694a22c44d45cb3fd34cf813e99)
\newline

Fixes: \url{https://github.com/oasis-tcs/virtio-spec/issues/183}\newline
Reviewed-by: Xuan Zhuo <xuanzhuo@linux.alibaba.com>\newline
Signed-off-by: Parav Pandit <parav@nvidia.com>\newline
Message-Id: <\nolinkurl{20240606102014.2103986-3-parav@nvidia.com}>\newline
Signed-off-by: Michael S. Tsirkin <mst@redhat.com>\newline
Signed-off-by: Matias Ezequiel Vara Larsen <mvaralar@redhat.com>\newline
Link: \url{https://lore.kernel.org/r/20250401133543.801184-8-mvaralar@redhat.com}\newline

 } \\
\hline
fb926ab38216 & 10 Apr 2025 & Parav Pandit & { virtio-net: Annotate virtio_net_hdr as field
\newline


At several places struct virtio_net_hdr missed out the field
annotation. Add it.
\newline

(cherry picked from commit 6abd42fd2398718ff689dc51fa93d38ede97be8f)
\newline

Reviewed-by: Xuan Zhuo <xuanzhuo@linux.alibaba.com>\newline
Signed-off-by: Parav Pandit <parav@nvidia.com>\newline
Message-Id: <\nolinkurl{20240606102014.2103986-4-parav@nvidia.com}>\newline
Signed-off-by: Michael S. Tsirkin <mst@redhat.com>\newline
Signed-off-by: Matias Ezequiel Vara Larsen <mvaralar@redhat.com>\newline
Link: \url{https://lore.kernel.org/r/20250401133543.801184-9-mvaralar@redhat.com}\newline

 } \\
\hline
b16f382569e1 & 10 Apr 2025 & Martin Kröning & { virtio_pci_cap64: specify offset_hi, length_hi endianness
\newline


While the capability introduction says "This virtio structure capability uses little-endian format,"
it might be preferrable to be explicit about the endianness of offset_hi and length_hi.
\newline

(cherry picked from commit 9b3129fe72360a78e76b6dd890d3abc5a45fa915)
\newline

Signed-off-by: Martin Kröning <martin.kroening@eonerc.rwth-aachen.de>\newline
Reviewed-by: Parav Pandit <parav@nvidia.com>\newline
Fixes: \url{https://github.com/oasis-tcs/virtio-spec/issues/196}\newline
Message-Id: <\nolinkurl{DF89BB0F-6BA4-4DAB-AEC3-03AAF858EC8E@eonerc.rwth-aachen.de}>\newline
Signed-off-by: Michael S. Tsirkin <mst@redhat.com>\newline
Signed-off-by: Matias Ezequiel Vara Larsen <mvaralar@redhat.com>\newline
Link: \url{https://lore.kernel.org/r/20250401133543.801184-10-mvaralar@redhat.com}\newline

 } \\
\hline
376348fe4047 & 10 Apr 2025 & Parav Pandit & { gpu: editorial: Fix spelling errors
\newline


Fix spelling errors.
\newline

(cherry picked from commit ec1845bd5a0261a65e29137f949ba03bf2fb44e2)
\newline

Branch: virtio-1.4\newline
Fixes: fed64230bf31 ("Add virtio gpu device specification.")\newline
Fixes: \url{https://github.com/oasis-tcs/virtio-spec/issues/205}\newline
Reviewed-by: Matias Ezequiel Vara Larsen <mvaralar@redhat.com>\newline
Signed-off-by: Parav Pandit <parav@nvidia.com>\newline
Link: \url{https://lore.kernel.org/r/20240916030835.68178-5-parav@nvidia.com}\newline
Signed-off-by: Matias Ezequiel Vara Larsen <mvaralar@redhat.com>\newline
Link: \url{https://lore.kernel.org/r/20250401133543.801184-11-mvaralar@redhat.com}\newline

 } \\
\hline
12289b0cf007 & 10 Apr 2025 & Paolo Abeni & { virtio-net: clarify NEEDS_CSUM semantic for GSO packats.
\newline


The current wording is a bit unclear hinting to possible additional
nested headers. For GSO packets virtio net (currently) supports
offload for the checksum of single transport header, explicitly state
that in both the driver and device sections.
\newline

(cherry picked from commit 8e443f483843c3d42a4128778f1c7548a02c48bf)
\newline

Signed-off-by: Paolo Abeni <pabeni@redhat.com>\newline
Reviewed-by: Jason Wang <jasowang@redhat.com>\newline
Reviewed-by: Parav Pandit <parav@nvidia.com>\newline
Link: \url{https://lore.kernel.org/r/81034d9f4b12f7bd7aa6f7f5266cb6d551d0823c.1732699986.git.pabeni@redhat.com}\newline
Signed-off-by: Matias Ezequiel Vara Larsen <mvaralar@redhat.com>\newline
Link: \url{https://lore.kernel.org/r/20250401133543.801184-12-mvaralar@redhat.com}\newline

 } \\
\hline
d3c2f14d3406 & 10 Apr 2025 & Paolo Abeni & { virtio-net: clarify DATA_VALID semantic for encap protos.
\newline


DATA_VALID allows offloading a single checksum level, leaving
unspecified which header checksum is offloaded when one or more
encapsulated protocols are present.
\newline

In such a case, the only option usable from the guest OS is
offloading the outermost checksum. That also matches the existing
implementation.
\newline

Explicitly state such the constraint, to remove any ambiguity and
make later changes more straightforward.
\newline

(cherry picked from commit 95c085b2f16de4bcdcae9c42d7828d9b8efc7836)
\newline

Signed-off-by: Paolo Abeni <pabeni@redhat.com>\newline
Reviewed-by: Jason Wang <jasowang@redhat.com>\newline
Reviewed-by: Parav Pandit <parav@nvidia.com>\newline
Link: \url{https://lore.kernel.org/r/ec6fdbfba2c4fc1969d83799836ebb694a01fb30.1732699986.git.pabeni@redhat.com}\newline
Signed-off-by: Matias Ezequiel Vara Larsen <mvaralar@redhat.com>\newline
Link: \url{https://lore.kernel.org/r/20250401133543.801184-13-mvaralar@redhat.com}\newline

 } \\
\hline
16718cb6912a & 10 Apr 2025 & Steffen Trumtrar & { virtio-net: Fix receive buffer size typo
\newline


The commit 00806815385340dd411cc67df3f6837935bb5e26 introduced a slight
typo in the struct virtio_net_hdr size calculation depending on
VIRTIO_NET_F_HASH_REPORT negotiation.
\newline

Without VIRTIO_NET_F_HASH_REPORT the struct is smaller than with the
feature. This mix up only occurs in one instance; sizes are correct in
all other occurences.
\newline

Fix this typo.
\newline

(cherry picked from commit 124fcd0e97f209aab19639e7369116d99ede22a2)
\newline

Fixes: 008068153853 ("virtio-net: Fix receive buffer size calculation text")\newline
Signed-off-by: Steffen Trumtrar <s.trumtrar@pengutronix.de>\newline
Reviewed-by: Parav Pandit <parav@nvidia.com>\newline
Acked-by: Michael S. Tsirkin <mst@redhat.com>\newline
Fixes: \url{https://github.com/oasis-tcs/virtio-spec/issues/216}\newline
Link: \url{https://lore.kernel.org/r/20250207-v1-4-topic-virtio-net-receive-buffer-fix-v1-1-efcef167d6bc@pengutronix.de}\newline
Signed-off-by: Matias Ezequiel Vara Larsen <mvaralar@redhat.com>\newline
Link: \url{https://lore.kernel.org/r/20250401133543.801184-14-mvaralar@redhat.com}\newline

 } \\
\hline
8d76f64d2198 & 20 May 2025 & Kommula Shiva Shankar & { virtio-net: Introduce a new field to indicate outer network header offset
\newline


This patch introduces a new field in the virtio_net_hdr called outer_nh_offset, along with a new net device feature, VIRTIO_NET_F_OUT_NET_HEADER.
\newline

Currently, drivers lack a dedicated field to signal the start of the network header to the device when performing checksum offload
and segmentation offload. This requires the device to read the packet in data path, which significantly affects performance.
Additionally, some hardware implementations require knowledge of the outer L3 offset (aka L2 length) for inline IPsec hardware acceleration.
\newline

To address this limitation, we propose to introduce a new field in the virtio_net_hdr called
outer_nh_offset.
\newline

The outer_nh_offset represents the start byte offset of the outer network header from the beginning of the packet.
\newline

This issue was briefly discussed on the mailing list in a different thread, which can be found here.
\url{https://lore.kernel.org/all/DM4PR18MB4269FAAC3CFC7E57E25DFBD2DF8B2@DM4PR18MB4269.namprd18.prod.outlook.com/}
\newline

v4->v5
 - Added padding bytes to virtio_net_hdr to ensure 64b alignment
 - Addressed pending review comments
v4:\url{https://lore.kernel.org/virtio-comment/20250304075955.208450-1-kshankar@marvell.com/}
\newline

v3 -> v4
 - Removed the union of new flag with existing flags. Added as a separate field
   in the virtio net header
 - Renamed out_nh_offset to outer_nh_offset to maintain consistency with other fields
 - Spellchecks in commit message description
v3:\url{https://lore.kernel.org/all/20250217172509.107212-1-kshankar@marvell.com}
\newline

v2 -> v3:
 - Rebase to virtio-1.4
 - Addressed pending review comments related to wording.
v2:\url{https://lore.kernel.org/all/20250128142152.3662988-1-kshankar@marvell.com/}
\newline

v1 -> v2:
 - explicitly state that the out_nh_offset can be set only when a valid network header is present.
 - updated out_nh_offset usage in the RX direction.
 - minor word cleanup.
v1: \url{https://lore.kernel.org/virtio-comment/20250114171636.3175670-1-kshankar@marvell.com/}
\newline

Signed-off-by: Kommula Shiva Shankar <kshankar@marvell.com>\newline
Reviewed-by: Parav Pandit <parav@nvidia.com>\newline
Fixes: \url{https://github.com/oasis-tcs/virtio-spec/issues/222}\newline
Link: \url{https://lore.kernel.org/r/20250401195655.486230-1-kshankar@marvell.com}\newline

 } \\
\hline
fa12149158bb & 20 May 2025 & Aiswarya Cyriac & { Reserve device ID 49 for Virtio USB controller device
\newline


Virtio USB controller device is a dual role device, which
can function as a USB host controller or a USB device
controller or support both roles. Additionally, device provides
support for switching of roles between USB host and device.
\newline

Signed-off-by: Aiswarya Cyriac <quic_acyriac@quicinc.com>\newline
Reviewed-by: Matias Ezequiel Vara Larsen <mvaralar@redhat.com>\newline
Fixes: \url{https://github.com/oasis-tcs/virtio-spec/issues/211}\newline
Link: \url{https://lore.kernel.org/r/20241129111325.952-1-quic_acyriac@quicinc.com}\newline

 } \\
\hline
c7553a71b7eb & 04 Jun 2025 & Srujana Challa & { virtio-crypto: Add IPsec service operation and Capabilities
\newline


This commit introduces the IPsec service operation to the Crypto
device, enabling offloading of IPsec processing.
\newline

Capabilities:\newline

1. IPsec Resource Capability (VIRTIO_CRYPTO_IPSEC_RESOURCE_CAP):
   Indicates the device's IPsec resource limits, such as the number of
   outbound and inbound Security Associations (SAs).
2. IPsec SA Capability (VIRTIO_CRYPTO_IPSEC_SA_CAP): Specifies the
   supported IPsec modes, along with the supported cryptographic
   algorithms, authentication algorithms, IPsec options and
   anti-replay window size.
\newline

Signed-off-by: Srujana Challa <schalla@marvell.com>\newline
Reviewed-by: Parav Pandit <parav@nvidia.com>\newline
Fixes: \url{https://github.com/oasis-tcs/virtio-spec/issues/226}\newline
Link: \url{https://lore.kernel.org/r/20250429131953.1949757-2-schalla@marvell.com}\newline

 } \\
\hline
9d2cf8dae5f7 & 04 Jun 2025 & Srujana Challa & { virtio-crypto: Add resource objects for IPsec outbound and inbound SAs
\newline


This commit introduces resource objects to enable the driver/device to
create IPsec Security Associations (SAs) for both inbound and outbound
directions.
\newline

The IPsec SA objects include essential parameters required for packet
outbound and inbound processing, such as SPI, tunnel headers, IPsec mode,
IPsec options and cipher/authentication specific data.
\newline

Signed-off-by: Srujana Challa <schalla@marvell.com>\newline
Reviewed-by: Parav Pandit <parav@nvidia.com>\newline
Fixes: \url{https://github.com/oasis-tcs/virtio-spec/issues/226}\newline
Link: \url{https://lore.kernel.org/r/20250429131953.1949757-3-schalla@marvell.com}\newline

 } \\
\hline
4e0baa89787f & 04 Jun 2025 & Srujana Challa & { virtio-crypto: Add new IPsec opcodes to data request
\newline


Adds new IPsec opcodes, VIRTIO_CRYPTO_IPSEC_OUTBOUND and
VIRTIO_CRYPTO_IPSEC_INBOUND and defines opcode specific
data structures for IPsec data processing.
\newline

Reviewed-by: Matias Ezequiel Vara Larsen <mvaralar@redhat.com>\newline
Reviewed-by: Parav Pandit <parav@nvidia.com>\newline
Signed-off-by: Srujana Challa <schalla@marvell.com>\newline
Fixes: \url{https://github.com/oasis-tcs/virtio-spec/issues/226}\newline
Link: \url{https://lore.kernel.org/r/20250429131953.1949757-4-schalla@marvell.com}\newline

 } \\
\hline
dac8e793bae1 & 04 Jun 2025 & Srujana Challa & { virtio-crypto: Add device and driver requirements for IPsec operation
\newline


Add device and driver requirements for IPsec Operation.
\newline

Signed-off-by: Srujana Challa <schalla@marvell.com>\newline
Reviewed-by: Parav Pandit <parav@nvidia.com>\newline
Fixes: \url{https://github.com/oasis-tcs/virtio-spec/issues/226}\newline
Link: \url{https://lore.kernel.org/r/20250429131953.1949757-5-schalla@marvell.com}\newline

 } \\
\hline
45809a3045e5 & 04 Jun 2025 & Srujana Challa & { virtio-net: Add IPsec operation, capabilities and resource objects
\newline


This commit introduces the IPsec Operation to the Net device
along with the capabilities and resource objects. This enables
the offloading of IPsec processing, both before transmission
and after reception, thereby providing inline offload
capabilities.
\newline

Capbilities:\newline

1. IPsec Resource Capability (VIRTIO_CRYPTO_IPSEC_RESOURCE_CAP):
   Indicates the device's IPsec resource limits, such as the number of
   encryption and decryption Security Associations (SAs).
2. IPsec SA Capability (VIRTIO_CRYPTO_IPSEC_SA_CAP): Specifies the
   supported IPsec modes, along with the supported cryptographic
   algorithms, authentication algorithms, IPsec options and
   anti-replay window size.
\newline

Resource objects:
1. VIRTIO_NET_RESOURCE_OBJ_IPSEC_OUTB_SA
2. VIRTIO_NET_RESOURCE_OBJ_IPSEC_INB_SA
\newline

These IPsec SA resource objects encompass parameters necessary
for packet encryption and decryption. These include the SPI,
tunnel headers, IPsec mode, IPsec options, and metadata specific
to cipher and authentication.
\newline

This patch refers the Virtio-crypto IPsec service operation
capabilities and resource objects data structures and crypto algorithm
definitions to avoid duplication, however the admin command type vaule
differs between Virtio-crypto and Virtio-net.
\newline

Signed-off-by: Srujana Challa <schalla@marvell.com>\newline
Reviewed-by: Parav Pandit <parav@nvidia.com>\newline
Fixes: \url{https://github.com/oasis-tcs/virtio-spec/issues/227}\newline
Link: \url{https://lore.kernel.org/r/20250520121924.2169258-2-schalla@marvell.com}\newline

 } \\
\hline
a9604face331 & 04 Jun 2025 & Srujana Challa & { virtio-net: Add new flow filter selector and action for IPsec
\newline


This update introduces a new flow filter selector to match
the ESP header and adds a new flow filter action for IPsec
processing.
\newline

Signed-off-by: Srujana Challa <schalla@marvell.com>\newline
Reviewed-by: Parav Pandit <parav@nvidia.com>\newline
Fixes: \url{https://github.com/oasis-tcs/virtio-spec/issues/227}\newline
Link: \url{https://lore.kernel.org/r/20250520121924.2169258-3-schalla@marvell.com}\newline

 } \\
\hline
fd15f89a870f & 04 Jun 2025 & Srujana Challa & { virtio-net: extend virtio_net_hdr for IPsec support
\newline


Add IPsec resource object identifiers to the virtio_net_hdr for
identifying encryption/decryption operations on tx and rx side
respectively, along with flags.
\newline

Signed-off-by: Srujana Challa <schalla@marvell.com>\newline
Reviewed-by: Parav Pandit <parav@nvidia.com>\newline
Fixes: \url{https://github.com/oasis-tcs/virtio-spec/issues/227}\newline
Link: \url{https://lore.kernel.org/r/20250520121924.2169258-4-schalla@marvell.com}\newline

 } \\
\hline
a0b809a7ddbd & 04 Jun 2025 & Srujana Challa & { virtio-net: Add IPsec operation device and driver requirements
\newline


Add device and driver requirements for IPsec Operation.
\newline

Signed-off-by: Srujana Challa <schalla@marvell.com>\newline
Reviewed-by: Parav Pandit <parav@nvidia.com>\newline
Fixes: \url{https://github.com/oasis-tcs/virtio-spec/issues/227}\newline
Link: \url{https://lore.kernel.org/r/20250520121924.2169258-5-schalla@marvell.com}\newline

 } \\
\hline
c5e5810cb6cb & 09 Jul 2025 & Albert Esteve & { virtio-media: Add virtio media device specification
\newline


Virtio-media is an encapsulation of the V4L2 UAPI into
virtio, able to virtualize any video device supported
by V4L2.
\newline

Note that virtio-media does not require the use of a
V4L2 device driver on the host or guest side -
V4L2 is only used as a host-guest protocol,
and both sides are free to convert it from/to any
model that they wish to use.
\newline

Reviewed-by: Matias Ezequiel Vara Larsen <mvaralar@redhat.com>\newline
Reviewed-by: Alexandre Courbot <gnurou@gmail.com>\newline
Signed-off-by: Albert Esteve <aesteve@redhat.com>\newline

 } \\
\hline
340d076eee2a & 15 Jul 2025 & Zhu Lingshan & { virtio: re-order device status bits
\newline


This commit re-arranges the device status bits,
to list them in ascending order.
\newline

Signed-off-by: Zhu Lingshan <lingshan.zhu@amd.com>\newline
Reviewed-by: Parav Pandit <parav@nvidia.com>\newline
Fixes: \url{https://github.com/oasis-tcs/virtio-spec/issues/229}\newline
Reviewed-by: Jason Wang <jasowang@redhat.com>\newline
Link: \url{https://lore.kernel.org/r/20250704101739.354522-2-lingshan.zhu@amd.com}\newline

 } \\
\hline
b1d6ef63b9f0 & 15 Jul 2025 & Zhu Lingshan & { virtio: document feature bit 42
\newline


This commit documents feture bit 42
VIRTIO_NET_F_GUEST_RSC6
\newline

Signed-off-by: Zhu Lingshan <lingshan.zhu@amd.com>\newline
Reviewed-by: Parav Pandit <parav@nvidia.com>\newline
Fixes: 94384142 ("content: Declare virtio-net legacy feature bits 41-42")\newline
Fixes: \url{https://github.com/oasis-tcs/virtio-spec/issues/229}\newline
Reviewed-by: Jason Wang <jasowang@redhat.com>\newline
Link: \url{https://lore.kernel.org/r/20250704101739.354522-3-lingshan.zhu@amd.com}\newline

 } \\
\hline
f219feffa995 & 15 Jul 2025 & Zhu Lingshan & { virtio: introduce SUSPEND and RESUME feature
\newline


This commit allows the driver to suspend the
device through a new device status bit SUSPEND
and resume the device running by re-setting
DRIVER_OK bit in device status.
\newline

Signed-off-by: Zhu Lingshan <lingshan.zhu@amd.com>\newline
Signed-off-by: Jason Wang <jasowang@redhat.com>\newline
Reviewed-by: Parav Pandit <parav@nvidia.com>\newline
Fixes: \url{https://github.com/oasis-tcs/virtio-spec/issues/229}\newline
Link: \url{https://lore.kernel.org/r/20250704101739.354522-4-lingshan.zhu@amd.com}\newline

 } \\
\hline
6275e115286b & 15 Jul 2025 & Parav Pandit & { virtio-net: Fix ipsec broken conformance links
\newline


Fix the broken conformance links for ipsec device and driver
requirements.
\newline

Fixes: a0b809a7ddbd ("virtio-net: Add IPsec operation device and driver requirements")\newline
Signed-off-by: Parav Pandit <parav@nvidia.com>\newline
Reviewed-by: Matias Ezequiel Vara Larsen <mvaralar@redhat.com>\newline
Link: \url{https://lore.kernel.org/r/20250709140049.507870-1-parav@nvidia.com}\newline

 } \\
\hline
65645f272406 & 22 Sep 2025 & Peter Hilber & { virtio-rtc: Add initial device specification
\newline


The virtio-rtc device provides information about current time through
one or more clocks. As such, it is a Real-Time Clock (RTC) device.
\newline

The normative statements for this device follow in the next patch.
\newline

For this device, there is a Linux kernel driver patch series which is
being upstreamed, and a proprietary device implementation.
\newline

Miscellaneous
\newline

-------------
\newline

The spec does not specify how a driver should interpret clock readings,
esp. also not how to perform clock synchronization.
\newline

The device uses the former "Timer/Clock" device id which is already part
of the specification. This device id was registered a long time ago and
should be unused according to the author's information. The name "RTC"
was determined to be the best for a device which focuses on current
time.
\newline

Signed-off-by: Peter Hilber <quic_philber@quicinc.com>\newline
Link: \url{https://lore.kernel.org/r/20250710090648.1711-3-quic_philber@quicinc.com}\newline

 } \\
\hline
f111987e3ff6 & 22 Sep 2025 & Peter Hilber & { virtio-rtc: Add initial normative statements
\newline


Add the normative statements for the initial device specification.
\newline

Signed-off-by: Peter Hilber <quic_philber@quicinc.com>\newline
Link: \url{https://lore.kernel.org/r/20250710090648.1711-4-quic_philber@quicinc.com}\newline

 } \\
\hline
0a6a441c3658 & 22 Sep 2025 & Peter Hilber & { virtio-rtc: Add alarm feature
\newline


Add the VIRTIO_RTC_F_ALARM feature (without normative statements).
\newline

The intended use case is: A driver needs to react when an alarm time has
been reached, but at alarm time, the driver may be in a sleep state or
powered off. The alarm feature can resume and notify the driver in this
case. Alarms may be retained across device resets.
\newline

Peculiarities
\newline

-------------
\newline

Unlike usual alarm clocks, a virtio-rtc alarm-capable clock may step
autonomously at any time: An alarm may change back from "expired" to
"not expired" before the driver has started processing an alarm
notification.
\newline

To address the above, and the device resets, define "alarm expiration"
in such a way that the driver always has a chance to react to an alarm,
and make the device always responsible for notifying the driver about an
alarm expiration.
\newline

The VIRTIO_RTC_REQ_SET_ALARM_ENABLED request is there so that the Linux
ioctls RTC_AIE_ON and RTC_AIE_OFF only need to emit one request.
\newline

Signed-off-by: Peter Hilber <quic_philber@quicinc.com>\newline
Link: \url{https://lore.kernel.org/r/20250710090648.1711-5-quic_philber@quicinc.com}\newline

 } \\
\hline
f978e0727353 & 22 Sep 2025 & Peter Hilber & { virtio-rtc: Add normative statements for alarm feature
\newline


Add the normative statements for the alarm feature added previously.
\newline

Signed-off-by: Peter Hilber <quic_philber@quicinc.com>\newline
Link: \url{https://lore.kernel.org/r/20250710090648.1711-6-quic_philber@quicinc.com}\newline

 } \\
\hline
928d969251bf & 23 Sep 2025 & Matias Ezequiel Vara Larsen & { git-publish: add profile
\newline


Add git-publish profile and document how to use it.
\newline

Signed-off-by: Matias Ezequiel Vara Larsen <mvaralar@redhat.com>\newline
Reviewed-by: Stefano Garzarella <sgarzare@redhat.com>\newline
Reviewed-by: Albert Esteve <aesteve@redhat.com>\newline
Acked-by: Cornelia Huck <cohuck@redhat.com>\newline
Link: \url{https://lore.kernel.org/r/20250305164546.1484029-1-mvaralar@redhat.com}\newline

 } \\
\hline
d9742cb7b26c & 26 Sep 2025 & Matias Ezequiel Vara Larsen & { README.md: add example email for TC vote request
\newline


Signed-off-by: Matias Ezequiel Vara Larsen <mvaralar@redhat.com>\newline
Link: \url{https://lore.kernel.org/r/20250320114326.2075821-1-mvaralar@redhat.com}\newline

 } \\
\hline
63aaa4bd96be & 13 Oct 2025 & Parav Pandit & { Merge branch 'virtio-1.4'
\newline


Resolved conflict in net device description.
\newline

Signed-off-by: Parav Pandit <parav@nvidia.com>\newline

 } \\
\hline
58ae15e84c02 & 28 Oct 2025 & Parav Pandit & { edit: remove old changelog
\newline


Prepare the changelog file for 1.4, so remove old change log and
move it to cl-cs05.tex.
\newline

Signed-off-by: Parav Pandit <parav@nvidia.com>\newline
Message-Id: <\nolinkurl{20251027174756.56284-2-parav@nvidia.com}>\newline

 } \\
\hline
9f903b20b702 & 28 Oct 2025 & Parav Pandit & { edit: add changelog for 1.4
\newline


Add the changelog for 1.4.
\newline

Signed-off-by: Parav Pandit <parav@nvidia.com>\newline
Message-Id: <\nolinkurl{20251027174756.56284-3-parav@nvidia.com}>\newline

 } \\
\hline
b4899d24cd09 & 28 Oct 2025 & Parav Pandit & { edit: Remove annotation of special character
\newline


PDF generation parsing complains when special character
annotation is listed in it. Remove such annotation in changelog.
\newline

Signed-off-by: Parav Pandit <parav@nvidia.com>\newline
Message-Id: <\nolinkurl{20251027174756.56284-4-parav@nvidia.com}>\newline

 } \\
\hline
5e4a575fc2d1 & 28 Oct 2025 & Michael S. Tsirkin & { drop generated files
\newline


Generated files:
    virtio-v1.3-csd01.aux
    virtio-v1.3-csd01.log
    virtio-v1.3-csd01.out
    virtio-v1.3-csd01.pdf
    virtio-v1.3-csd01.toc
\newline

have no business being in tree. Drop them.
\newline

Fixes: 63aaa4b ("Merge branch 'virtio-1.4'")\newline
Cc: "Parav Pandit" <parav@nvidia.com>\newline
Signed-off-by: Michael S. Tsirkin <mst@redhat.com>\newline

 } \\
\hline
21e81faef360 & 06 Nov 2025 & Michael S. Tsirkin & { virtio-html: add missing makeatother
\newline


A workaround for F21 added \textbackslash makeatletter but not \textbackslash makeatother
\newline

I see no specific issues around this but theoretically can interfere
with some packages.
\newline

Fix this up.
\newline

Fixes: 86e51b4 ("html: work around bug in html generation")\newline
Signed-off-by: Michael S. Tsirkin <mst@redhat.com>\newline

 } \\
\hline
563b6c8abb59 & 07 Nov 2025 & Michael S. Tsirkin & { editorial: suppress noitem errors in diff
\newline


latexdiff tends to leave empty itemize/enumerate lists around
instead of deleting lists.
\newline

For example:
\newline

\textbackslash begin\{itemize\}\%DIFAUXCMD
\%DIFDELCMD <   \textbackslash item rx_usecs: Maximum number of usecs to delay a RX notification.
\%DIFDELCMD <
\%DIFDELCMD <   \textbackslash item rx_max_packets: Maximum number of packets to receive before a RX notification.
\textbackslash end\{itemize\}\%DIFAUXCMD
\newline

this makes latex stop with an error:
    Something's wrong--perhaps a missing \textbackslash item
to fix, suppress the error.
\newline

Signed-off-by: Michael S. Tsirkin <mst@redhat.com>\newline

 } \\
\hline
f9abfd55cb66 & 07 Nov 2025 & Sergio Lopez & { virtio-gpu: support blob alignment information
\newline


There's an increasing number of machines supporting multiple page sizes
and, on these machines, the host and a guest can be running with
different pages sizes. In addition to this, there might be physical
devices that require to operate their memory at a specific granularity.
\newline

In these cases, if they are to use Shared Memory Regions, the device
and the driver must operate with the same granularity, as otherwise
the former might not be able to fulfill the requests sent by the
latter.
\newline

For the GPU device, this has an impact on blob creation and mapping. To
address the problem, allow the device to require certain alignment
constrains for blob resources by extending the device configuration
with the field "blob_alignment" and introducing the
VIRTIO_GPU_F_BLOB_ALIGNMENT feature.
\newline

Signed-off-by: Sergio Lopez <slp@redhat.com>\newline

 } \\
\hline
b2741ed3bfa9 & 07 Nov 2025 & Parav Pandit & { virtio-gpu: Fix missing driver conformance file
\newline


Cited patch in the fixes tag missed to include the driver-conformance
file. Add it.
\newline

Fixes: f52fb20cbe38 ("virtio-gpu: support blob alignment information")\newline
Signed-off-by: Parav Pandit <parav@nvidia.com>\newline
Reviewed-by: Matias Ezequiel Vara Larsen <mvaralar@redhat.com>\newline
Link: \url{https://lore.kernel.org/r/20250614181302.277794-1-parav@nvidia.com}\newline

 } \\
\hline
8494bb93333f & 07 Nov 2025 & Parav Pandit & { virtio-crypto: editorial: fix broken label link
\newline


Fix broken link to resource objects.
\newline

Fixes: c7553a71b7eb ("virtio-crypto: Add IPsec service operation and Capabilities")\newline
Reported-by: Michael S. Tsirkin <mst@redhat.com>\newline
Signed-off-by: Parav Pandit <parav@nvidia.com>\newline

 } \\
\hline
e8557d8d3360 & 07 Nov 2025 & Parav Pandit & { changelog: Add changelog for 1.2 and 1.3
\newline


Add the changelog from version 1.2 to v1.3 for bookkeeping purpose.
\newline

Signed-off-by: Parav Pandit <parav@nvidia.com>\newline

 } \\
\hline
521d5cccc608 & 07 Nov 2025 & Parav Pandit & { gitlog: Add executable permision to script
\newline


gitlog.pl is the script to generate the changelog.
Make it executable.
\newline

Signed-off-by: Parav Pandit <parav@nvidia.com>\newline

 } \\
\hline
4e2a6b93be64 & 07 Nov 2025 & Parav Pandit & { acknowledgements: update for 1.4
\newline


Move some names to the section for previous versions, add names of new
contributors, etc.
\newline

Signed-off-by: Parav Pandit <parav@nvidia.com>\newline

 } \\
\hline
0bad6c8dc8b7 & 07 Nov 2025 & Parav Pandit & { changelog: Update changelog for additonal 1.4 patches
\newline


Update the change log further for missing ipsec, gpu blob
and other editorial patches.
\newline

Signed-off-by: Parav Pandit <parav@nvidia.com>\newline

 } \\
\hline
14b4403e30c5 & 10 Nov 2025 & Parav Pandit & { virtio-net: editorial: fix broken label link
\newline


Fix broken link to crypto outbound SA resource objects.
\newline

Fixes: fd15f89a870f ("virtio-net: extend virtio_net_hdr for IPsec support")\newline
Reported-by: Michael S. Tsirkin <mst@redhat.com>\newline
Signed-off-by: Parav Pandit <parav@nvidia.com>\newline

 } \\
\hline
9e065a90dd47 & 11 Nov 2025 & Parav Pandit & { editorial: Add a new chair and a new editor
\newline


Add Matias as the chair an Parav as the enditor.
\newline

Signed-off-by: Parav Pandit <parav@nvidia.com>\newline

 } \\
\hline
0d71ab85e916 & 11 Nov 2025 & Parav Pandit & { editorial: Update the new oasis open link
\newline


OASIS updated the link to the virtio tc community lately.
Update the link to it.
\newline

Signed-off-by: Parav Pandit <parav@nvidia.com>\newline

 } \\
\hline
9191b1e7e27b & 13 Nov 2025 & Parav Pandit & { changelog: Update changelog for additonal 1.4 patches
\newline


Update the change log further for acknowledgement and other
editorial patches.
\newline

Signed-off-by: Parav Pandit <parav@nvidia.com>\newline

 } \\
\hline
7672c71eeb65 & 13 Nov 2025 & Parav Pandit & { REVISION: update to 1.4
\newline


Update the revision to 1.4 working draft and date.
Since 1.3 was never released, keep the last version of 1.2 for diff generation.
\newline

Signed-off-by: Parav Pandit <parav@nvidia.com>\newline

 } \\
\hline
f6c548ff7182 & 16 Nov 2025 & Parav Pandit & { editorial: Remove hyperlink with hash tag
\newline


The new hyperlink contains the hash letter.
The new replaced link breaks the diff generation.
\newline

This is because diff operation generates DIFadd, DIFdel
commands inside the href.
\newline

This causes deep recursive macro expansion resulting into
below error.
Avoid this small change in the diff generation.
\newline

Remove the other patches which already exists now in the
master branch.
\newline

! TeX capacity exceeded, sorry [input stack size=10000].
\newline

\textbackslash @setfontsize \#1\#2\#3->\textbackslash @nomath \#1 \textbackslash ifx \textbackslash protect
\newline

\textbackslash @typeset@protect \textbackslash let \textbackslash @curr... l.577 ...delines/tc-process\#wpComponentsCompLang\}.
\newline


Signed-off-by: Parav Pandit <parav@nvidia.com>\newline

 } \\
\hline
92a4614ac9f7 & 16 Nov 2025 & Parav Pandit & { editorial: Avoid logtable keyword as diff
\newline


Commit X started using longtable.
Diff generation is sneaking logtable between DIFaddbegin
and DIFaddend. This makes LaTeX unhappy breaking the
diff generation.
\newline

Using SAFEENV or PICTUREENV or appending texcmd or appending
text cmd is not helpful either to resolve to generate the
desired result. It deletes the whole table and adds as
new table showing big delta, which is undesired.
\newline

Until diff infrastructure improved, workaround to remove those
weird diffs with simpler perl script.
\newline

Signed-off-by: Parav Pandit <parav@nvidia.com>\newline

 } \\
\hline
267ced17df4a & 16 Nov 2025 & Parav Pandit & { editorial: Update diff version to 1.2
\newline


Update the previous version to 1.2 so that 1.4 spec
diff can be with previously release version 1.2.
\newline

Signed-off-by: Parav Pandit <parav@nvidia.com>\newline

 } \\
\hline
5cc223827632 & 16 Nov 2025 & Parav Pandit & { changelog: Update changelog for additonal 1.4 patches
\newline


Update the change log further for editorial patches related
to diff generation with previously released version 1.2.
\newline

Signed-off-by: Parav Pandit <parav@nvidia.com>\newline

 } \\
\hline
fc3c135083f4 & 16 Nov 2025 & Parav Pandit & { editorial: Advance the date with fixes for diff generation
\newline


Advance the date for editorial fixes addition for
diff generation.
\newline

Signed-off-by: Parav Pandit <parav@nvidia.com>\newline

 } \\
\hline
7cdd4521c657 & 18 Nov 2025 & Parav Pandit & { editorial: Update revision to csd01
\newline


Update the revision to v1.4-csd01 and its new date.
\newline

Signed-off-by: Parav Pandit <parav@nvidia.com>\newline

 } \\
\hline
75e806852781 & 18 Nov 2025 & Parav Pandit & { editorial: Prepare public review draft
\newline


Prepare public review draft version 1.4-csprd01.
\newline

Signed-off-by: Parav Pandit <parav@nvidia.com>\newline

 } \\
\hline
26c96dd4b684 & 19 Nov 2025 & Parav Pandit & { editorial: Restore back the previous stage links
\newline


Restore back the link to the previous version 1.2.
\newline

Signed-off-by: Parav Pandit <parav@nvidia.com>\newline

 } \\
\hline
defdc1bf3679 & 19 Nov 2025 & Parav Pandit & { Revert "editorial: Prepare public review draft"
\newline


This reverts commit 75e806852781ae98035df6f97c7af748eff1e9c1.
Few issues were found in the draft.
Hence, revert to restart the process.
\newline

Signed-off-by: Parav Pandit <parav@nvidia.com>\newline

 } \\
\hline
5833db4fbcf1 & 19 Nov 2025 & Parav Pandit & { Revert "editorial: Update revision to csd01"
\newline


This reverts commit 7cdd4521c657e893489b7722fb693d936743cb36.
Few issues were found in the draft.
Hence, revert to restart the process.
\newline

Signed-off-by: Parav Pandit <parav@nvidia.com>\newline

 } \\
\hline
2132a93d51b7 & 03 Dec 2025 & Michael S. Tsirkin & { net: pad virtio_net_ff_cap_data
\newline


struct virtio_net_ff_cap_data has 4 byte fields but the size is not a
multiple of 4.  drivers can easily get it wrong since compilers tend to
add padding to align such structures.
\newline

Since we are always allowed to pad or truncate admin commands, let's do
just that here.
\newline

Fixes: 899bb0ca24d8 ("virtio-net: Add flow filter capability")\newline
Fixes: \url{https://github.com/oasis-tcs/virtio-spec/issues/236}\newline
Cc: Parav Pandit <parav@nvidia.com>\newline
Signed-off-by: Michael S. Tsirkin <mst@redhat.com>\newline
Reviewed-by: Parav Pandit <parav@nvidia.com>\newline

 } \\
\hline
53f06a6c432a & 03 Dec 2025 & Parav Pandit & { changelog: Update changelog for additonal 1.4 patches
\newline


Update the change log capability structure padding and
other editorial changes.
\newline

Signed-off-by: Parav Pandit <parav@nvidia.com>\newline

 } \\
\hline
b7f4aed22b11 & 03 Dec 2025 & Parav Pandit & { editorial: Advance revision date
\newline


Advance the revision date to cover for the capability pad fix.
\newline

Signed-off-by: Parav Pandit <parav@nvidia.com>\newline

 } \\
\hline
760fc4b7f19a & 03 Dec 2025 & Parav Pandit & { editorial: Update revision to csd01
\newline


Update the revision to v1.4-csd01.
\newline

Signed-off-by: Parav Pandit <parav@nvidia.com>\newline

 } \\
\hline
373609d6cbe2 & 03 Dec 2025 & Parav Pandit & { editorial: Prepare public review draft
\newline


Prepare public review draft version 1.4-csprd01.
\newline

Signed-off-by: Parav Pandit <parav@nvidia.com>\newline

 } \\
\hline
89d81932dc92 & 09 Dec 2025 & Michael S. Tsirkin & { edit: Add script to add newline to changelog
\newline


Add a helper script file which adds newline tag to the
changelog file to make them more readable in the revision
history.
\newline

Signed-off-by: Michael S. Tsirkin <mst@redhat.com>\newline
Signed-off-by: Parav Pandit <parav@nvidia.com>\newline

 } \\
\hline
