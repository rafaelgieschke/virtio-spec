44eb958fbd78 & 19 Apr 2022 & Cornelia Huck & { Revert "iommu: offered -> negotiated"
\newline


This reverts commit 071d283887352e5353e3d212fbd672861b1e550b.
\newline

This change needs more discussion, and proper voting.
\newline

Signed-off-by: Cornelia Huck <cohuck@redhat.com>\newline

 } \\
\hline
02105f0bf900 & 19 Apr 2022 & Michael S. Tsirkin & { packed-ring: fix some typos
\newline


The VIRTQ_DESC_F_INDIRECT flag is misnamed in a couple of places.
\newline

Signed-off-by: Michael S. Tsirkin <mst@redhat.com>\newline
Signed-off-by: Cornelia Huck <cohuck@redhat.com>\newline

 } \\
\hline
a987975583fd & 20 Apr 2022 & Michael S. Tsirkin & { packed-ring.tex: link conformance statements
\newline


Link conformance statements into conformance chapter.
\newline

Signed-off-by: Michael S. Tsirkin <mst@redhat.com>\newline
Signed-off-by: Cornelia Huck <cohuck@redhat.com>\newline

 } \\
\hline
81a54ad91fdb & 20 Apr 2022 & Michael S. Tsirkin & { content.tex: drop space after \textbackslash field
\newline


Always use \textbackslash field\{foo\} not \textbackslash field \{foo\}, the latter confuses
latexdiff.
\newline

Signed-off-by: Michael S. Tsirkin <mst@redhat.com>\newline
Signed-off-by: Cornelia Huck <cohuck@redhat.com>\newline

 } \\
\hline
7c5cbf957b1b & 20 Apr 2022 & Michael S. Tsirkin & { fixupdiff: escape caret
\newline


Since commit:
\newline

commit 1e30753d53d222fbe955f0643583d756105d557e
Author: Jan Kiszka <jan.kiszka@siemens.com>\newline
Date:   Fri Oct 11 13:43:41 2019 +0200\newline

    Trying to escaping escape caret here only leaves the backslash in the output.
\newline

    Signed-off-by: Jan Kiszka <jan.kiszka@siemens.com>
    Signed-off-by: Michael S. Tsirkin <mst@redhat.com>
\newline

we have plain escape caret inside listing. Escape it with textbackslash  but only
if not already escaped (as output of diff with old source will be).
\newline

Signed-off-by: Michael S. Tsirkin <mst@redhat.com>\newline
Signed-off-by: Cornelia Huck <cohuck@redhat.com>\newline

 } \\
\hline
5da7c1414e7e & 13 Jun 2022 & Stefan Hajnoczi & { virtio-blk: document that the capacity field can change
\newline


Block devices can change size during operation. A configuration change
notification is sent by the device and the driver detects that the field
has changed. Document this behavior that has already been implemented in
Linux and QEMU since 2011.
\newline

Fixes: \url{https://github.com/oasis-tcs/virtio-spec/issues/136}\newline

Signed-off-by: Stefan Hajnoczi <stefanha@redhat.com>\newline
Signed-off-by: Cornelia Huck <cohuck@redhat.com>\newline

 } \\
\hline
ad2e1674bb69 & 13 Jun 2022 & Laura Loghin & { vsock: add documentation about len header field
\newline


Fixes: \url{https://github.com/oasis-tcs/virtio-spec/issues/137}\newline

Reviewed-by: Stefano Garzarella <sgarzare@redhat.com>\newline
Signed-off-by: Laura Loghin <lauralg@amazon.com>\newline
Signed-off-by: Cornelia Huck <cohuck@redhat.com>\newline

 } \\
\hline
fca015771bc9 & 13 Jun 2022 & Xuan Zhuo & { virtio-net: support reset queue
\newline


A separate reset queue function introduced by Virtqueue Reset.
\newline

However, it is currently not defined what to do if the destination queue is
being reset when virtio-net is steering in multi-queue mode.
\newline

Fixes: \url{https://github.com/oasis-tcs/virtio-spec/issues/138}\newline

Reviewed-by: Jason Wang <jasowang@redhat.com>\newline
Signed-off-by: Xuan Zhuo <xuanzhuo@linux.alibaba.com>\newline
Signed-off-by: Cornelia Huck <cohuck@redhat.com>\newline

 } \\
\hline
6328f51e21b5 & 24 Jun 2022 & Yuri Benditovich & { virtio-net: define guest USO features
\newline


Fixes: \url{https://github.com/oasis-tcs/virtio-spec/issues/120}\newline

Add definition for large UDP packets device-to-driver.
\newline

Signed-off-by: Yuri Benditovich <yuri.benditovich@daynix.com>\newline
Signed-off-by: Cornelia Huck <cohuck@redhat.com>\newline

 } \\
\hline
49ff7805924c & 24 Jun 2022 & Anton Yakovlev & { virtio-snd: add support for audio controls
\newline


This patch extends the virtio sound device specification by adding
support for audio controls. Audio controls can be used to set the volume
level, mute/unmute the audio signal, switch different modes/states of
the virtual sound device, etc.
\newline

Fixes: \url{https://github.com/oasis-tcs/virtio-spec/issues/107}\newline

Signed-off-by: Anton Yakovlev <anton.yakovlev@opensynergy.com>\newline
Signed-off-by: Cornelia Huck <cohuck@redhat.com>\newline

 } \\
\hline
4d9068effa81 & 11 Jul 2022 & Alvaro Karsz & { Introduction of Virtio Network device notifications coalescing feature.
\newline


Control a network device notifications coalescing parameters using the control virtqueue.
A new control class was added: VIRTIO_NET_CTRL_NOTF_COAL.
\newline

This class provides 2 commands:
\newline

- VIRTIO_NET_CTRL_NOTF_COAL_TX_SET:
  Ask the network device to change the tx_usecs and tx_max_packets parameters.
  - tx_usecs: Maximum number of usecs to delay a TX notification.
  - tx_max_packets: Maximum number of packets to send before a TX notification.
\newline


- VIRTIO_NET_CTRL_NOTF_COAL_RX_SET:
  Ask the network device to change the rx_usecs and rx_max_packets parameters.
  - rx_usecs: Maximum number of usecs to delay a RX notification.
  - rx_max_packets: Maximum number of packets to receive before a RX notification.
\newline

Fixes: \url{https://github.com/oasis-tcs/virtio-spec/issues/141}\newline

Reviewed-by: Jason Wang <jasowang@redhat.com>\newline
Signed-off-by: Alvaro Karsz <alvaro.karsz@solid-run.com>\newline
[CH: fixed commit message]
Signed-off-by: Cornelia Huck <cohuck@redhat.com>\newline

 } \\
\hline
685a46bd33dd & 20 Jul 2022 & Cornelia Huck & { Merge branch 'virtio-1.3'
\newline


Signed-off-by: Cornelia Huck <cohuck@redhat.com>\newline

 } \\
\hline
abbe8afda8db & 03 Aug 2022 & Lei He & { virtio-crypto: introduce akcipher service
\newline


Introduce akcipher (asymmetric key cipher) service type, several
asymmetric algorithms and relevent information:
  - RSA(padding algorithm, ASN.1 schema definition)
  - ECDSA(ECC algorithm)
\newline

Fixes: \url{https://github.com/oasis-tcs/virtio-spec/issues/129}\newline

Signed-off-by: zhenwei pi <pizhenwei@bytedance.com>\newline
Signed-off-by: Lei He <helei.sig11@bytedance.com>\newline
Signed-off-by: Cornelia Huck <cohuck@redhat.com>\newline

 } \\
\hline
26ed30ccb049 & 03 Aug 2022 & Stefano Garzarella & { virtio-vsock: add VIRTIO_VSOCK_F_NO_IMPLIED_STREAM feature bit
\newline


Initially virtio-vsock only supported the stream type, which is why
there was no feature. Later we added the seqpacket type and in the future
we may have other types (e.g. datagram).
\newline

seqpacket is an extension of stream, so it might be implied that if
seqpacket is supported, stream is too, but this might not be true for
other types.
\newline

As we discussed here [1] should be better to add a new
VIRTIO_VSOCK_F_NO_IMPLIED_STREAM feature bit to avoid this implication.
\newline

Let's also add normative sections to better define the behavior when
VIRTIO_VSOCK_F_NO_IMPLIED_STREAM is negotiated or not.
\newline

[1] \url{http://markmail.org/message/2s3qd74drgjxkvte}
\newline

Fixes: \url{https://github.com/oasis-tcs/virtio-spec/issues/142}\newline

Suggested-by: Michael S. Tsirkin <mst@redhat.com>\newline
Acked-by: Michael S. Tsirkin <mst@redhat.com>\newline
Signed-off-by: Stefano Garzarella <sgarzare@redhat.com>\newline
Signed-off-by: Cornelia Huck <cohuck@redhat.com>\newline

 } \\
\hline
a7251b0cb4d9 & 14 Nov 2022 & Hrishivarya Bhageeradhan & { content: reserve device ID 43 for Camera device
\newline


The virtio-camera device allows to stream a camera video with
ability to change controls, formats and get camera captures.
This patch is to reserve the next available device ID for
virtio-camera.
\newline

Fixes: \url{https://github.com/oasis-tcs/virtio-spec/issues/148}\newline

Signed-off-by: Hrishivarya Bhageeradhan <hrishivarya.bhageeradhan@opensynergy.com>\newline
Signed-off-by: Cornelia Huck <cohuck@redhat.com>\newline

 } \\
\hline
b4e8efa0fa6c & 05 Dec 2022 & Dmitry Fomichev & { virtio-blk: add zoned block device specification
\newline


Introduce support for Zoned Block Devices to virtio.
\newline

Zoned Block Devices (ZBDs) aim to achieve a better capacity, latency
and/or cost characteristics compared to commonly available block
devices by getting the entire LBA space of the device divided to block
regions that are much larger than the LBA size. These regions are
called zones and they can only be written sequentially. More details
about ZBDs can be found at
\newline

\url{https://zonedstorage.io/docs/introduction/zoned-storage} .
\newline

In its current form, the virtio protocol for block devices (virtio-blk)
is not aware of ZBDs but it allows the driver to successfully scan a
host-managed drive provided by the virtio block device. As the result,
the host-managed drive is recognized by virtio driver as a regular,
non-zoned drive that will operate erroneously under the most common
write workloads. Host-aware ZBDs are currently usable, but their
performance may not be optimal because the driver can only see them as
non-zoned block devices.
\newline

To fix this, the virtio-blk protocol needs to be extended to add the
capabilities to convey the zone characteristics of ZBDs at the device
side to the driver and to provide support for ZBD-specific commands -
Report Zones, four zone operations (Open, Close, Finish and Reset) and
(optionally) Zone Append. The proposed standard extension aims to
define this new functionality.
\newline

This patch extends the virtio-blk section of virtio specification with
the minimum set of requirements that are necessary to support ZBDs.
The resulting device model is a subset of the models defined in ZAC/ZBC
and ZNS standards documents. The included functionality mirrors
the existing Linux kernel block layer ZBD support and should be
sufficient to handle the host-managed and host-aware HDDs that are on
the market today as well as ZNS SSDs that are entering the market at
the time of submission of this patch.
\newline

I would like to thank the following people for their useful feedback
and suggestions while working on the initial iterations of this patch.
\newline

Damien Le Moal <damien.lemoal@opensource.wdc.com>
Matias Bjørling <Matias.Bjorling@wdc.com>
Niklas Cassel <Niklas.Cassel@wdc.com>
Hans Holmberg <Hans.Holmberg@wdc.com>
\newline

Fixes: \url{https://github.com/oasis-tcs/virtio-spec/issues/143}\newline

Signed-off-by: Dmitry Fomichev <dmitry.fomichev@wdc.com>\newline
Reviewed-by: Stefan Hajnoczi <stefanha@redhat.com>\newline
Reviewed-by: Damien Le Moal <damien.lemoal@opensource.wdc.com>\newline
Signed-off-by: Cornelia Huck <cohuck@redhat.com>\newline

 } \\
\hline
985bbf397db4 & 07 Dec 2022 & Xuan Zhuo & { content: reserve device ID 44 for ISM device
\newline


The virtio-ism device provides the ability to share memory between
different guests on a host. A guest's memory got from ism device can be
shared with multiple peers at the same time. This shared relationship
can be dynamically created and released.
\newline

The shared memory obtained from the device is divided into multiple ism
regions for share. ISM device provides a mechanism to notify other ism
region referrers of content update events.
\newline

This patch is to reserve the next available device ID for virtio-ism.
\newline

Fixes: \url{https://github.com/oasis-tcs/virtio-spec/issues/150}\newline

Signed-off-by: Xuan Zhuo <xuanzhuo@linux.alibaba.com>\newline
Signed-off-by: Jiang Liu <gerry@linux.alibaba.com>\newline
Signed-off-by: Dust Li <dust.li@linux.alibaba.com>\newline
Signed-off-by: Tony Lu <tonylu@linux.alibaba.com>\newline
Signed-off-by: Helin Guo <helinguo@linux.alibaba.com>\newline
Signed-off-by: Hans Zhang <hans@linux.alibaba.com>\newline
Signed-off-by: He Rongguang <herongguang@linux.alibaba.com>\newline
Signed-off-by: Cornelia Huck <cohuck@redhat.com>\newline

 } \\
\hline
f2b28698a28a & 30 Jan 2023 & Parav Pandit & { virtio-net: Maintain network device spec in separate directory
\newline


Move virtio network device specification to its own file similar to
recent virtio devices.
While at it, place device specification, its driver and device
conformance into its own directory to have self contained device
specification.
\newline

Fixes: \url{https://github.com/oasis-tcs/virtio-spec/issues/153}\newline

Acked-by: Michael S. Tsirkin <mst@redhat.com>\newline
Signed-off-by: Parav Pandit <parav@nvidia.com>\newline
Signed-off-by: Cornelia Huck <cohuck@redhat.com>\newline

 } \\
\hline
81694cddc4c1 & 30 Jan 2023 & Parav Pandit & { virtio-net: Fix spelling errors
\newline


Fix two spelling errors in the virtio network device specification.
\newline

Acked-by: Michael S. Tsirkin <mst@redhat.com>\newline
Signed-off-by: Parav Pandit <parav@nvidia.com>\newline
Signed-off-by: Cornelia Huck <cohuck@redhat.com>\newline

 } \\
\hline
335342f5cd88 & 30 Jan 2023 & Parav Pandit & { virtio-blk: Maintain block device spec in separate directory
\newline


Move virtio block device specification to its own file similar to
recent virtio devices.
While at it, place device specification, its driver and device
conformance into its own directory to have self contained device
specification.
\newline

Fixes: \url{https://github.com/oasis-tcs/virtio-spec/issues/153}\newline

Acked-by: Michael S. Tsirkin <mst@redhat.com>\newline
Signed-off-by: Parav Pandit <parav@nvidia.com>\newline
Signed-off-by: Cornelia Huck <cohuck@redhat.com>\newline

 } \\
\hline
d3d06187eabb & 30 Jan 2023 & Parav Pandit & { virtio-console: Maintain console device spec in separate directory
\newline


Move virtio console device specification to its own file similar to
recent virtio devices.
While at it, place device specification, its driver and device
conformance into its own directory to have self contained device
specification.
\newline

Fixes: \url{https://github.com/oasis-tcs/virtio-spec/issues/153}\newline

Acked-by: Michael S. Tsirkin <mst@redhat.com>\newline
Signed-off-by: Parav Pandit <parav@nvidia.com>\newline
Signed-off-by: Cornelia Huck <cohuck@redhat.com>\newline

 } \\
\hline
c71e88e86d35 & 30 Jan 2023 & Parav Pandit & { virtio-entropy: Maintain entropy device spec in separate directory
\newline


Move virtio entropy device specification to its own file similar to
recent virtio devices.
While at it, place device specification, its driver and device
conformance into its own directory to have self contained device
specification.
\newline

Fixes: \url{https://github.com/oasis-tcs/virtio-spec/issues/153}\newline

Acked-by: Michael S. Tsirkin <mst@redhat.com>\newline
Signed-off-by: Parav Pandit <parav@nvidia.com>\newline
Signed-off-by: Cornelia Huck <cohuck@redhat.com>\newline

 } \\
\hline
c06f3b670dd6 & 30 Jan 2023 & Parav Pandit & { virtio-balloon: Maintain mem balloon device spec in separate directory
\newline


Move virtio memory balloon device specification to its own file
similar to recent virtio devices.
While at it, place device specification, its driver and device
conformance into its own directory to have self contained device
specification.
\newline

Fixes: \url{https://github.com/oasis-tcs/virtio-spec/issues/153}\newline

Acked-by: Michael S. Tsirkin <mst@redhat.com>\newline
Signed-off-by: Parav Pandit <parav@nvidia.com>\newline
Signed-off-by: Cornelia Huck <cohuck@redhat.com>\newline

 } \\
\hline
d404f1c4e886 & 30 Jan 2023 & Parav Pandit & { virtio-scsi: Maintain scsi host device spec in separate directory
\newline


Move virtio SCSI host device specification to its own file similar to
recent virtio devices.
While at it, place device specification, its driver and device
conformance into its own directory to have self contained device
specification.
\newline

Fixes: \url{https://github.com/oasis-tcs/virtio-spec/issues/153}\newline

Acked-by: Michael S. Tsirkin <mst@redhat.com>\newline
Signed-off-by: Parav Pandit <parav@nvidia.com>\newline
Signed-off-by: Cornelia Huck <cohuck@redhat.com>\newline

 } \\
\hline
442bb643a9ad & 30 Jan 2023 & Parav Pandit & { virtio-gpu: Maintain gpu device spec in separate directory
\newline


Move virtio gpu device specification to its own file similar to
recent virtio devices.
While at it, place device specification, its driver and device
conformance into its own directory to have self contained device
specification.
\newline

Fixes: \url{https://github.com/oasis-tcs/virtio-spec/issues/153}\newline

Acked-by: Michael S. Tsirkin <mst@redhat.com>\newline
Signed-off-by: Parav Pandit <parav@nvidia.com>\newline
Signed-off-by: Cornelia Huck <cohuck@redhat.com>\newline

 } \\
\hline
c9686f241819 & 30 Jan 2023 & Parav Pandit & { virtio-input: Maintain input device spec in separate directory
\newline


Move virtio input device specification to its own file similar to
recent virtio devices.
While at it, place device specification, its driver and device
conformance into its own directory to have self contained device
specification.
\newline

Fixes: \url{https://github.com/oasis-tcs/virtio-spec/issues/153}\newline

Acked-by: Michael S. Tsirkin <mst@redhat.com>\newline
Signed-off-by: Parav Pandit <parav@nvidia.com>\newline
Signed-off-by: Cornelia Huck <cohuck@redhat.com>\newline

 } \\
\hline
8463bba27c79 & 30 Jan 2023 & Parav Pandit & { virtio-crypto: Maintain crypto device spec in separate directory
\newline


Move virtio crypto device specification to its own file similar to
recent virtio devices.
While at it, place device specification, its driver and device
conformance into its own directory to have self contained device
specification.
\newline

Fixes: \url{https://github.com/oasis-tcs/virtio-spec/issues/153}\newline

Acked-by: Michael S. Tsirkin <mst@redhat.com>\newline
Signed-off-by: Parav Pandit <parav@nvidia.com>\newline
Signed-off-by: Cornelia Huck <cohuck@redhat.com>\newline

 } \\
\hline
828754b98e3b & 30 Jan 2023 & Parav Pandit & { virtio-vsock: Maintain socket device spec in separate directory
\newline


Place device specification, its driver and device
conformance into its own directory to have self contained device
specification.
\newline

Fixes: \url{https://github.com/oasis-tcs/virtio-spec/issues/153}\newline

Acked-by: Michael S. Tsirkin <mst@redhat.com>\newline
Reviewed-by: Stefano Garzarella <sgarzare@redhat.com>\newline
Signed-off-by: Parav Pandit <parav@nvidia.com>\newline
Signed-off-by: Cornelia Huck <cohuck@redhat.com>\newline

 } \\
\hline
8632f80e251f & 30 Jan 2023 & Parav Pandit & { virtio-fs: Maintain file system device spec in separate directory
\newline


Place device specification, its driver and device
conformance into its own directory to have self contained device
specification.
\newline

Fixes: \url{https://github.com/oasis-tcs/virtio-spec/issues/153}\newline

Acked-by: Michael S. Tsirkin <mst@redhat.com>\newline
Signed-off-by: Parav Pandit <parav@nvidia.com>\newline
Signed-off-by: Cornelia Huck <cohuck@redhat.com>\newline

 } \\
\hline
b067de47a506 & 30 Jan 2023 & Parav Pandit & { virtio-rpmb: Maintain rpmb device spec in separate directory
\newline


Place device specification, its driver and device
conformance into its own directory to have self contained device
specification.
\newline

Fixes: \url{https://github.com/oasis-tcs/virtio-spec/issues/153}\newline

Acked-by: Michael S. Tsirkin <mst@redhat.com>\newline
Signed-off-by: Parav Pandit <parav@nvidia.com>\newline
Signed-off-by: Cornelia Huck <cohuck@redhat.com>\newline

 } \\
\hline
b1cf73e96173 & 30 Jan 2023 & Parav Pandit & { virtio-iommu: Maintain iommu device spec in separate directory
\newline


Place device specification, its driver and device
conformance into its own directory to have self contained device
specification.
\newline

Fixes: \url{https://github.com/oasis-tcs/virtio-spec/issues/153}\newline

Acked-by: Michael S. Tsirkin <mst@redhat.com>\newline
Signed-off-by: Parav Pandit <parav@nvidia.com>\newline
Signed-off-by: Cornelia Huck <cohuck@redhat.com>\newline

 } \\
\hline
6813e3cc271e & 30 Jan 2023 & Parav Pandit & { virtio-sound: Maintain sound device spec in separate directory
\newline


Place device specification, its driver and device
conformance into its own directory to have self contained device
specification.
\newline

Fixes: \url{https://github.com/oasis-tcs/virtio-spec/issues/153}\newline

Acked-by: Michael S. Tsirkin <mst@redhat.com>\newline
Signed-off-by: Parav Pandit <parav@nvidia.com>\newline
Signed-off-by: Cornelia Huck <cohuck@redhat.com>\newline

 } \\
\hline
5042a5031502 & 30 Jan 2023 & Parav Pandit & { virtio-mem: Maintain memory device spec in separate directory
\newline


Place device specification, its driver and device
conformance into its own directory to have self contained device
specification.
\newline

Fixes: \url{https://github.com/oasis-tcs/virtio-spec/issues/153}\newline

Acked-by: Michael S. Tsirkin <mst@redhat.com>\newline
Signed-off-by: Parav Pandit <parav@nvidia.com>\newline
Signed-off-by: Cornelia Huck <cohuck@redhat.com>\newline

 } \\
\hline
00b9935238bf & 30 Jan 2023 & Parav Pandit & { virtio-i2c: Maintain i2c device spec in separate directory
\newline


Place device specification, its driver and device
conformance into its own directory to have self contained device
specification.
\newline

Fixes: \url{https://github.com/oasis-tcs/virtio-spec/issues/153}\newline

Acked-by: Michael S. Tsirkin <mst@redhat.com>\newline
Signed-off-by: Parav Pandit <parav@nvidia.com>\newline
Signed-off-by: Cornelia Huck <cohuck@redhat.com>\newline

 } \\
\hline
674489b191ab & 30 Jan 2023 & Parav Pandit & { virtio-scmi: Maintain scmi device spec in separate directory
\newline


Place device specification, its driver and device
conformance into its own directory to have self contained device
specification.
\newline

Fixes: \url{https://github.com/oasis-tcs/virtio-spec/issues/153}\newline

Acked-by: Michael S. Tsirkin <mst@redhat.com>\newline
Signed-off-by: Parav Pandit <parav@nvidia.com>\newline
Signed-off-by: Cornelia Huck <cohuck@redhat.com>\newline

 } \\
\hline
6c9c04d2bf5e & 30 Jan 2023 & Parav Pandit & { virtio-gpio: Maintain gpio device spec in separate directory
\newline


Place device specification, its driver and device
conformance into its own directory to have self contained device
specification.
\newline

Fixes: \url{https://github.com/oasis-tcs/virtio-spec/issues/153}\newline

Acked-by: Michael S. Tsirkin <mst@redhat.com>\newline
Signed-off-by: Parav Pandit <parav@nvidia.com>\newline
Signed-off-by: Cornelia Huck <cohuck@redhat.com>\newline

 } \\
\hline
d04d253b1055 & 30 Jan 2023 & Parav Pandit & { virtio-pmem: Maintain pmem device spec in separate directory
\newline


Place device specification, its driver and device
conformance into its own directory to have self contained device
specification.
\newline

Fixes: \url{https://github.com/oasis-tcs/virtio-spec/issues/153}\newline

Acked-by: Michael S. Tsirkin <mst@redhat.com>\newline
Signed-off-by: Parav Pandit <parav@nvidia.com>\newline
Signed-off-by: Cornelia Huck <cohuck@redhat.com>\newline

 } \\
\hline
b1fb6b62495f & 02 Feb 2023 & Parav Pandit & { virtio-net: Clarify VLAN filter table configuration
\newline


The filtering behavior of the VLAN filter commands is not very clear as
discussed in thread [1].
\newline

Hence, add the command description and device requirements for it.
\newline

[1] \url{https://lists.oasis-open.org/archives/virtio-dev/202301/msg00210.html}
\newline

Fixes: \url{https://github.com/oasis-tcs/virtio-spec/issues/147}\newline

Suggested-by: Si-Wei Liu <si-wei.liu@oracle.com>\newline
Reviewed-by: Si-Wei Liu <si-wei.liu@oracle.com>\newline
Acked-by: Michael S. Tsirkin <mst@redhat.com>\newline
Signed-off-by: Parav Pandit <parav@nvidia.com>\newline
Signed-off-by: Cornelia Huck <cohuck@redhat.com>\newline

 } \\
\hline
53b0cb13169c & 02 Feb 2023 & Parav Pandit & { virtio-net: Avoid confusing device configuration text
\newline


The added text in commit of Fixes tag was redundant and
confusing in context of VLAN filtering description.
\newline

Hence remove it as discussed in [1] and [2].
\newline

[1] \url{https://lists.oasis-open.org/archives/virtio-dev/202301/msg00282.html}
[2] \url{https://lists.oasis-open.org/archives/virtio-dev/202301/msg00286.html}
\newline

Fixes: 296303444f6b ("virtio-net: Clarify VLAN filter table configuration")\newline

Suggested-by: Halil Pasic <pasic@linux.ibm.com>\newline
Acked-by: Michael S. Tsirkin <mst@redhat.com>\newline
Signed-off-by: Parav Pandit <parav@nvidia.com>\newline
[CH: applied as editorial change]
Signed-off-by: Cornelia Huck <cohuck@redhat.com>\newline

 } \\
\hline
3b9b6acb0936 & 09 Feb 2023 & Michael S. Tsirkin & { audio->sound
\newline


Spec calls the device "sound device". Make the name in the
ID section match.
\newline

MST: applied as editorial change.\newline

Signed-off-by: Michael S. Tsirkin <mst@redhat.com>\newline
Reviewed-by: Cornelia Huck <cohuck@redhat.com>\newline

 } \\
\hline
0ce03bc6995a & 14 Feb 2023 & Parav Pandit & { virtio-net: Avoid confusion between a card and a device
\newline


Historically virtio network device is documented as an Ethernet card.
A modern card in the industry has one to multiple ports, one to multiple
PCI functions. However the virtio network device is usually just a
single link/port network interface controller.
\newline

Hence, avoid this confusing term 'card' and align the specification
to adhere to widely used specification term as 'device' used for all
virtio device types.
\newline

Replaced 'card' with 'network interface controller'.
\newline

Fixes: \url{https://github.com/oasis-tcs/virtio-spec/issues/154}\newline

Signed-off-by: Parav Pandit <parav@nvidia.com>\newline
Signed-off-by: Cornelia Huck <cohuck@redhat.com>\newline

 } \\
\hline
be2ce1ee17e0 & 15 Feb 2023 & Parav Pandit & { content.tex Fix Driver notifications label
\newline


Driver notifications section is under "Basic Facilities of a Virtio
Device". However, the label is placed under "Virtqueues" section.
\newline

Fix the label references.
\newline

Acked-by: Michael S. Tsirkin <mst@redhat.com>\newline
Signed-off-by: Parav Pandit <parav@nvidia.com>\newline
[CH: pushed as an editorial update]
Signed-off-by: Cornelia Huck <cohuck@redhat.com>\newline

 } \\
\hline
2ea4627093fb & 20 Feb 2023 & Alvaro Karsz & { virtio-net: Mention VIRTIO_NET_F_HASH_REPORT dependency on VIRTIO_NET_F_CTRL_VQ
\newline


If the VIRTIO_NET_F_HASH_REPORT feature is negotiated, the driver may
send VIRTIO_NET_CTRL_MQ_HASH_CONFIG commands, thus, the control VQ
feature should be negotiated.
\newline

Fixes: \url{https://github.com/oasis-tcs/virtio-spec/issues/158}\newline
Signed-off-by: Alvaro Karsz <alvaro.karsz@solid-run.com>\newline
Signed-off-by: Cornelia Huck <cohuck@redhat.com>\newline

 } \\
\hline
73ce5bb02003 & 01 Mar 2023 & Alvaro Karsz & { virtio-net: Fix and update VIRTIO_NET_F_NOTF_COAL feature
\newline


This patch makes several improvements to the notification coalescing
feature, including:
\newline


- Consolidating virtio_net_ctrl_coal_tx and virtio_net_ctrl_coal_rx
  into a single struct, virtio_net_ctrl_coal, as they are identical.
\newline

- Emphasizing that the coalescing commands are best-effort.
\newline

- Defining the behavior of coalescing with regards to delivering
  notifications when a change occur.
\newline

- Stating that the commands should apply to all the receive/transmit
  virtqueues.
\newline

- Stating that every receive/transmit virtqueue should count it's own
  packets.
\newline

- A new intro explaining the entire coalescing operation.
\newline

Fixes: \url{https://github.com/oasis-tcs/virtio-spec/issues/159}\newline

Signed-off-by: Alvaro Karsz <alvaro.karsz@solid-run.com>\newline
Reviewed-by: Parav Pandit <parav@nvidia.com>\newline
Acked-by: Michael S. Tsirkin <mst@redhat.com>\newline
Signed-off-by: Cornelia Huck <cohuck@redhat.com>\newline

 } \\
\hline
3508347769af & 01 Mar 2023 & Parav Pandit & { virtio-net: Improve introductory description
\newline


The control VQ of the virtio network device is used beyond advance
steering control. The control VQ dynamically changes multiple features
of the initialized device.
\newline

Hence, update this area of control VQ introductory description at few
places and also place the link to its description.
\newline

Also update the introduction section to better describe receive and
transmit virtqueues.
\newline

Fixes: \url{https://github.com/oasis-tcs/virtio-spec/issues/156}\newline

Reviewed-by: David Edmondson <david.edmondson@oracle.com>\newline
Signed-off-by: Parav Pandit <parav@nvidia.com>\newline
Signed-off-by: Cornelia Huck <cohuck@redhat.com>\newline

 } \\
\hline
91a469991433 & 10 Mar 2023 & Parav Pandit & { transport-pci: Split PCI transport to its own file
\newline


Place PCI transport specification in its own file to better maintain it.
\newline

Fixes: \url{https://github.com/oasis-tcs/virtio-spec/issues/157}\newline

Signed-off-by: Parav Pandit <parav@nvidia.com>\newline
Signed-off-by: Cornelia Huck <cohuck@redhat.com>\newline

 } \\
\hline
9e88ba9c47d0 & 10 Mar 2023 & Parav Pandit & { transport-mmio: Split MMIO transport to its own file
\newline


Place MMIO transport specification in its own file to better maintain it.
\newline

Fixes: \url{https://github.com/oasis-tcs/virtio-spec/issues/157}\newline

Signed-off-by: Parav Pandit <parav@nvidia.com>\newline
Signed-off-by: Cornelia Huck <cohuck@redhat.com>\newline

 } \\
\hline
0af264f9d4ea & 10 Mar 2023 & Parav Pandit & { transport-ccw: Split Channel IO transport to its own file
\newline


Place Channel IO transport specification in its own file to
better maintain it.
\newline

Fixes: \url{https://github.com/oasis-tcs/virtio-spec/issues/157}\newline

Signed-off-by: Parav Pandit <parav@nvidia.com>\newline
Signed-off-by: Cornelia Huck <cohuck@redhat.com>\newline

 } \\
\hline
deb0aa0c7faa & 10 Mar 2023 & Parav Pandit & { transport-pci: Fix spellings and white spaces
\newline


Now that we have individual files, fix reported spelling errors.
\newline

While at it, remove trailing white spaces.
\newline

Fixes: \url{https://github.com/oasis-tcs/virtio-spec/issues/157}\newline

Signed-off-by: Parav Pandit <parav@nvidia.com>\newline
Signed-off-by: Cornelia Huck <cohuck@redhat.com>\newline

 } \\
\hline
ca97719ea35e & 10 Mar 2023 & Parav Pandit & { transport-mmio: Fix spellings and white spaces
\newline


Now that we have individual files, fix reported spelling errors.
\newline

While at it, remove trailing white spaces.
\newline

Fixes: \url{https://github.com/oasis-tcs/virtio-spec/issues/157}\newline

Signed-off-by: Parav Pandit <parav@nvidia.com>\newline
Signed-off-by: Cornelia Huck <cohuck@redhat.com>\newline

 } \\
\hline
8797f4d4e410 & 10 Mar 2023 & Parav Pandit & { transport-ccw: Fix spellings and white spaces
\newline


Now that we have individual files, fix reported spelling errors.
\newline

While at it, remove extra white spaces.
\newline

Fixes: \url{https://github.com/oasis-tcs/virtio-spec/issues/157}\newline

Signed-off-by: Parav Pandit <parav@nvidia.com>\newline
Signed-off-by: Cornelia Huck <cohuck@redhat.com>\newline

 } \\
\hline
d3f832b6605d & 15 Mar 2023 & Parav Pandit & { virtio-net: Describe dev cfg fields read only
\newline


Device configuration fields are read only. Avoid duplicating this
description for multiple fields.
\newline

Instead describe it one time and do it in the driver requirements
section.
\newline

Fixes: \url{https://github.com/oasis-tcs/virtio-spec/issues/161}\newline

Reviewed-by: David Edmondson <david.edmondson@oracle.com>\newline
Signed-off-by: Parav Pandit <parav@nvidia.com>\newline
Signed-off-by: Cornelia Huck <cohuck@redhat.com>\newline

 } \\
\hline
115ceb97f813 & 15 Mar 2023 & Parav Pandit & { virtio-net: Define cfg fields before description
\newline


Currently some fields of the virtio_net_config structure are defined
before introducing the structure and some are defined after.
Better to define the configuration layout first followed by
description of all the fields.
\newline

Device configuration fields are described in the section. Change wording
from 'listed' to 'described' as suggested in patch [1].
\newline

[1] \url{https://lists.oasis-open.org/archives/virtio-dev/202302/msg00004.html}
\newline

Fixes: \url{https://github.com/oasis-tcs/virtio-spec/issues/161}\newline

Reviewed-by: David Edmondson <david.edmondson@oracle.com>\newline
Signed-off-by: Parav Pandit <parav@nvidia.com>\newline
Signed-off-by: Cornelia Huck <cohuck@redhat.com>\newline

 } \\
\hline
2d1d8dfa3474 & 15 Mar 2023 & Parav Pandit & { virtio-net: Fix virtqueues spelling error
\newline


Correct spelling from virtqueus to virtqueues.
\newline

Signed-off-by: Parav Pandit <parav@nvidia.com>\newline
Acked-by: Michael S. Tsirkin <mst@redhat.com>\newline
Reviewed-by: Jiri Pirko <jiri@nvidia.com>\newline
[CH: pushed as editorial update]
Signed-off-by: Cornelia Huck <cohuck@redhat.com>\newline

 } \\
\hline
2d5495083c12 & 15 Mar 2023 & Parav Pandit & { transport-pci: Remove duplicate word structure
\newline


Remove duplicate word structure.
\newline

Signed-off-by: Parav Pandit <parav@nvidia.com>\newline
Acked-by: Michael S. Tsirkin <mst@redhat.com>\newline
Reviewed-by: Halil Pasic <pasic@linux.ibm.com>\newline
Reviewed-by: Jiri Pirko <jiri@nvidia.com>\newline
[CH: pushed as editorial update]
Signed-off-by: Cornelia Huck <cohuck@redhat.com>\newline

 } \\
\hline
f6fe1647aab6 & 15 Mar 2023 & Michael S. Tsirkin & { makediff: make it work for fresh checkout
\newline


1st time one checks out our repo, latexdiff submodule
is not initialized. Pass --init to update command
to initialize it. It seems to be harmless if already
initialized.
\newline

Signed-off-by: Michael S. Tsirkin <mst@redhat.com>\newline
Acked-by: Cornelia Huck <cohuck@redhat.com>\newline
Tested-by: Parav Pandit <parav@nvidia.com>\newline

 } \\
\hline
b0414098602f & 15 Mar 2023 & Parav Pandit & { virtio-blk: Define dev cfg layout before its fields
\newline


Define device configuration layout structure before describing its
individual fields.
\newline

This is an editorial change.
\newline

Suggested-by: Cornelia Huck <cohuck@redhat.com>\newline
Reviewed-by: Max Gurtovoy <mgurtovoy@nvidia.com>\newline
Signed-off-by: Parav Pandit <parav@nvidia.com>\newline
Signed-off-by: Michael S. Tsirkin <mst@redhat.com>\newline
Reviewed-by: Stefan Hajnoczi <stefanha@redhat.com>\newline

 } \\
\hline
380ed02bdb88 & 04 Apr 2023 & Parav Pandit & { transport-pci: Remove empty line at end of file
\newline


Remove empty line at end of file.
\newline

Signed-off-by: Parav Pandit <parav@nvidia.com>\newline
Signed-off-by: Michael S. Tsirkin <mst@redhat.com>\newline
Reviewed-by: David Edmondson <david.edmondson@oracle.com>\newline

 } \\
\hline
1ed0754c6134 & 11 Apr 2023 & Heng Qi & { virtio-net: support the virtqueue coalescing moderation
\newline


Currently, coalescing parameters are grouped for all transmit and receive
virtqueues. This patch supports setting or getting the parameters for a
specified virtqueue, and a typical application of this function is netdim[1].
\newline

When the traffic between virtqueues is unbalanced, for example, one virtqueue
is busy and another virtqueue is idle, then it will be very useful to
control coalescing parameters at the virtqueue granularity.
\newline

[1] \url{https://docs.kernel.org/networking/net_dim.html}
\newline

Fixes: \url{https://github.com/oasis-tcs/virtio-spec/issues/166}\newline

Signed-off-by: Heng Qi <hengqi@linux.alibaba.com>\newline
Reviewed-by: Xuan Zhuo <xuanzhuo@linux.alibaba.com>\newline
Reviewed-by: Parav Pandit <parav@nvidia.com>\newline
Signed-off-by: Cornelia Huck <cohuck@redhat.com>\newline

 } \\
\hline
362ebd007271 & 11 Apr 2023 & Alvaro Karsz & { virtio-net: define the VIRTIO_NET_F_CTRL_RX_EXTRA feature bit
\newline


The VIRTIO_NET_F_CTRL_RX_EXTRA feature bit is mentioned in the spec
since version 1.0, but it's not properly defined.
\newline

This patch defines the feature bit and defines the dependency on VIRTIO_NET_F_CTRL_VQ.
\newline

Since this dependency is missing in previous versions, we add it now as
a "SHOULD".
\newline

Fixes: \url{https://github.com/oasis-tcs/virtio-spec/issues/162}\newline

Reviewed-by: Parav Pandit <parav@nvidia.com>\newline
Signed-off-by: Alvaro Karsz <alvaro.karsz@solid-run.com>\newline
Signed-off-by: Cornelia Huck <cohuck@redhat.com>\newline

 } \\
\hline
d3b2a19bc369 & 21 Apr 2023 & Parav Pandit & { device-types/multiple: replace queues with enqueues
\newline


Queue is a verb and noun both. Replacing it with enqueue avoids
ambiguity around plural queues noun vs verb; similar to virtio fs device
description.
\newline

Acked-by: Michael S. Tsirkin <mst@redhat.com>\newline
Signed-off-by: Parav Pandit <parav@nvidia.com>\newline
[CH: pushed as editorial update]
Signed-off-by: Cornelia Huck <cohuck@redhat.com>\newline

 } \\
\hline
aadefe688680 & 19 May 2023 & Michael S. Tsirkin & { virtio: document forward compatibility guarantees
\newline


Feature negotiation forms the basis of forward compatibility
guarantees of virtio but has never been properly documented.
Do it now.
\newline

Suggested-by: Halil Pasic <pasic@linux.ibm.com>\newline
Signed-off-by: Michael S. Tsirkin <mst@redhat.com>\newline
Reviewed-by: Parav Pandit <parav@nvidia.com>\newline
Reviewed-by: Zhu Lingshan <lingshan.zhu@intel.com>\newline

 } \\
\hline
f3ce853c8a91 & 19 May 2023 & Michael S. Tsirkin & { admin: introduce device group and related concepts
\newline


Each device group has a type. For now, define one initial group type:
\newline

SR-IOV type - PCI SR-IOV virtual functions (VFs) of a given
PCI SR-IOV physical function (PF). This group may contain zero or more
virtio devices according to NumVFs configured.
\newline

Each device within a group has a unique identifier. This identifier
is the group member identifier.
\newline

Note: one can argue both ways whether the new device group handling\newline
functionality (this and following patches) is closer
to a new device type or a new transport type.
\newline

However, it's expected that we will add more features in the near
future. To facilitate this as much as possible of the text is located in
the new admin chapter.
\newline

Effort was made to minimize transport-specific text.
\newline

There's a bit of duplication with 0x1 repeated twice and
no special section for group type identifiers.
It seems ok to defer adding these until we have more group
types.
\newline

Signed-off-by: Michael S. Tsirkin <mst@redhat.com>\newline
Reviewed-by: Stefan Hajnoczi <stefanha@redhat.com>\newline

 } \\
\hline
2cbaaa19b15a & 19 May 2023 & Michael S. Tsirkin & { admin: introduce group administration commands
\newline


This introduces a general structure for group administration commands,
used to control device groups through their owner.
\newline

Following patches will introduce specific commands and an interface for
submitting these commands to the owner.
\newline

Note that the commands are focused on controlling device groups:
this is why group related fields are in the generic part of
the structure.
Without this the admin vq would become a "whatever" vq which does not do
anything specific at all, just a general transport like thing.
I feel going this way opens the design space to the point where
we no longer know what belongs in e.g. config space
what in the control q and what in the admin q.
As it is, whatever deals with groups is in the admin q; other
things not in the admin q.
\newline

There are specific exceptions such as query but that's an exception that
proves the rule ;)
\newline

Signed-off-by: Michael S. Tsirkin <mst@redhat.com>\newline
Reviewed-by: Stefan Hajnoczi <stefanha@redhat.com>\newline
Reviewed-by: Zhu Lingshan <lingshan.zhu@intel.com>\newline

 } \\
\hline
5f1a8ac61c15 & 19 May 2023 & Michael S. Tsirkin & { admin: introduce virtio admin virtqueues
\newline


The admin virtqueues will be the first interface used to issue admin commands.
\newline

Currently the virtio specification defines control virtqueue to manipulate
features and configuration of the device it operates on:
virtio-net, virtio-scsi, etc all have existing control virtqueues. However,
control virtqueue commands are device type specific, which makes it very
difficult to extend for device agnostic commands.
\newline

Keeping the device-specific virtqueue separate from the admin virtqueue
is simpler and has fewer potential problems. I don't think creating
common infrastructure for device-specific control virtqueues across
device types worthwhile or within the scope of this patch series.
\newline

To support this requirement in a more generic way, this patch introduces
a new admin virtqueue interface.
The admin virtqueue can be seen as the virtqueue analog to a transport.
The admin queue thus does nothing device type-specific (net, scsi, etc)
and instead focuses on transporting the admin commands.
\newline

We also support more than one admin virtqueue, for QoS and
scalability requirements.
\newline

Signed-off-by: Michael S. Tsirkin <mst@redhat.com>\newline
Reviewed-by: Stefan Hajnoczi <stefanha@redhat.com>\newline

 } \\
\hline
677aeaebf6a7 & 19 May 2023 & Michael S. Tsirkin & { pci: add admin vq registers to virtio over pci
\newline


Add new registers to the PCI common configuration structure.
\newline

These registers will be used for querying the indices of the admin
virtqueues of the owner device. To configure, reset or enable the admin
virtqueues, the driver should follow existing queue configuration/setup
sequence.
\newline

Signed-off-by: Michael S. Tsirkin <mst@redhat.com>\newline
Reviewed-by: Parav Pandit <parav@nvidia.com>\newline
Reviewed-by: Zhu Lingshan <lingshan.zhu@intel.com>\newline

 } \\
\hline
a9a59f70be46 & 19 May 2023 & Michael S. Tsirkin & { mmio: document ADMIN_VQ as reserved
\newline


Adding relevant registers needs more work and it's not
clear what the use-case will be as currently only
the PCI transport is supported. But let's keep the
door open on this.
We already say it's reserved in a central place, but it
does not hurt to remind implementers to mask it.
\newline

Signed-off-by: Michael S. Tsirkin <mst@redhat.com>\newline
Reviewed-by: Parav Pandit <parav@nvidia.com>\newline
Reviewed-by: Stefan Hajnoczi <stefanha@redhat.com>\newline

 } \\
\hline
325046c1460e & 19 May 2023 & Michael S. Tsirkin & { ccw: document ADMIN_VQ as reserved
\newline


Adding relevant registers needs more work and it's not
clear what the use-case will be as currently only
the PCI transport is supported. But let's keep the
door open on this.
We already say it's reserved in a central place, but it
does not hurt to remind implementers to mask it.
\newline

Note: there are more features to add to this list.\newline
Will be done later with a patch on top.
\newline

Signed-off-by: Michael S. Tsirkin <mst@redhat.com>\newline
Reviewed-by: Stefan Hajnoczi <stefanha@redhat.com>\newline
Reviewed-by: Parav Pandit <parav@nvidia.com>\newline
Reviewed-by: Zhu Lingshan <lingshan.zhu@intel.com>\newline

 } \\
\hline
3dc7196cba2d & 19 May 2023 & Michael S. Tsirkin & { admin: command list discovery
\newline


Add commands to find out which commands does each group support,
as well as enable their use by driver.
This will be especially useful once we have multiple group types.
\newline

An alternative is per-type VQs. This is possible but will
require more per-transport work. Discovery through the vq
helps keep things contained.
\newline

e.g. lack of support for some command can switch to a legacy mode
\newline

note that commands are expected to be avolved by adding new
fields to command specific data at the tail, so
we generally do not need feature bits for compatibility.
\newline

Signed-off-by: Michael S. Tsirkin <mst@redhat.com>\newline
Reviewed-by: Stefan Hajnoczi <stefanha@redhat.com>\newline
Reviewed-by: Zhu Lingshan <lingshan.zhu@intel.com>\newline

 } \\
\hline
bf1d6b0d24ae & 19 May 2023 & Michael S. Tsirkin & { admin: conformance clauses
\newline


Add conformance clauses for admin commands and admin virtqueues.
\newline

Fixes: \url{https://github.com/oasis-tcs/virtio-spec/issues/171}\newline
Signed-off-by: Michael S. Tsirkin <mst@redhat.com>\newline
Reviewed-by: Stefan Hajnoczi <stefanha@redhat.com>\newline

 } \\
\hline
b04be31f0bf0 & 19 May 2023 & Michael S. Tsirkin & { ccw: document more reserved features
\newline


vq reset and shared memory are unsupported, too.
\newline

Signed-off-by: Michael S. Tsirkin <mst@redhat.com>\newline
Fixes: \url{https://github.com/oasis-tcs/virtio-spec/issues/160}\newline
Reviewed-by: Stefan Hajnoczi <stefanha@redhat.com>\newline
Reviewed-by: Zhu Lingshan <lingshan.zhu@intel.com>\newline

 } \\
\hline
619f60ae4ccf & 19 May 2023 & Parav Pandit & { admin: Fix reference and table formation
\newline


This patch brings three fixes.
\newline

1. Opcode table has 3 columns, only two were enumerated. Due to this
pdf generation script stops. Fix it and also have resizeable description
column as it needs wrap.
\newline

2. Status description column content needs to wrap. Without it pdf
   does not read good. Fix it by having resizeable description column.
\newline

3. Fix the broken link to the Device groups.
\newline

Fixes: 2cbaaa1 ("admin: introduce group administration commands")\newline
Signed-off-by: Parav Pandit <parav@nvidia.com>\newline
Signed-off-by: Michael S. Tsirkin <mst@redhat.com>\newline
Reviewed-by: Cornelia Huck <cohuck@redhat.com>\newline

 } \\
\hline
c1cd68b97611 & 19 May 2023 & Parav Pandit & { transport-pci: Improve config msix vector description
\newline


config_msix_vector is the register that holds the MSI-X vector number
for receiving configuration change related interrupts.
\newline

It is not "for MSI-X".
\newline

Hence, replace the confusing text with appropriate one.
\newline

Fixes: \url{https://github.com/oasis-tcs/virtio-spec/issues/169}\newline
Reviewed-by: Max Gurtovoy <mgurtovoy@nvidia.com>\newline
Signed-off-by: Parav Pandit <parav@nvidia.com>\newline
Signed-off-by: Michael S. Tsirkin <mst@redhat.com>\newline

 } \\
\hline
0f433d62e81d & 19 May 2023 & Parav Pandit & { transport-pci: Improve queue msix vector register desc
\newline


queue_msix_vector register is for receiving virtqueue notification
interrupts from the device for the virtqueue.
\newline

"for MSI-X" is confusing term.
\newline

Also it is the register that driver "writes" to, similar to
many other registers such as queue_desc, queue_driver etc.
\newline

Hence, replace the verb from use to write.
\newline

Fixes: \url{https://github.com/oasis-tcs/virtio-spec/issues/169}\newline
Signed-off-by: Parav Pandit <parav@nvidia.com>\newline
Reviewed-by: Max Gurtovoy <mgurtovoy@nvidia.com>\newline
Signed-off-by: Michael S. Tsirkin <mst@redhat.com>\newline

 } \\
\hline
b0fbccd4062f & 19 May 2023 & Parav Pandit & { content: Add vq index text
\newline


Introduce vq index and its range so that subsequent patches can refer
to it.
\newline

Fixes: \url{https://github.com/oasis-tcs/virtio-spec/issues/163}\newline
Reviewed-by: David Edmondson <david.edmondson@oracle.com>\newline
Reviewed-by: Halil Pasic <pasic@linux.ibm.com>\newline
Signed-off-by: Parav Pandit <parav@nvidia.com>\newline
Signed-off-by: Michael S. Tsirkin <mst@redhat.com>\newline

 } \\
\hline
362f1cac2516 & 19 May 2023 & Parav Pandit & { content.tex Replace virtqueue number with index
\newline


Replace virtqueue number with index to align to rest of the
specification.
\newline

Fixes: \url{https://github.com/oasis-tcs/virtio-spec/issues/163}\newline
Reviewed-by: David Edmondson <david.edmondson@oracle.com>\newline
Reviewed-by: Halil Pasic <pasic@linux.ibm.com>\newline
Signed-off-by: Parav Pandit <parav@nvidia.com>\newline
Signed-off-by: Michael S. Tsirkin <mst@redhat.com>\newline

 } \\
\hline
cc4a5604b259 & 19 May 2023 & Parav Pandit & { content: Rename confusing queue_notify_data and vqn names
\newline


Currently queue_notify_data register indicates the device
internal queue notification content. This register is
meaningful only when feature bit VIRTIO_F_NOTIF_CONFIG_DATA is
negotiated.
\newline

However, above register name often get confusing association with
very similar feature bit VIRTIO_F_NOTIFICATION_DATA.
\newline

When VIRTIO_F_NOTIFICATION_DATA feature bit is negotiated,
notification really involves sending additional queue progress
related information (not queue identifier or index).
\newline

Hence
1. to avoid any misunderstanding and association of
queue_notify_data with similar name VIRTIO_F_NOTIFICATION_DATA,
\newline

and
2. to reflect that queue_notify_data is the actual device
internal virtqueue identifier/index/data/cookie,
\newline

a. rename queue_notify_data to queue_notif_config_data.
\newline

b. rename ambiguous vqn to a union of vq_index and vq_config_data
\newline

c. The driver notification section assumes that queue notification contains
vq index always. CONFIG_DATA feature bit introduction missed to
update the driver notification section. Hence, correct it.
\newline

Fixes: \url{https://github.com/oasis-tcs/virtio-spec/issues/163}\newline
Acked-by: Halil Pasic <pasic@linux.ibm.com>\newline
Signed-off-by: Parav Pandit <parav@nvidia.com>\newline
Signed-off-by: Michael S. Tsirkin <mst@redhat.com>\newline

Reviewed-by: David Edmondson <david.edmondson@oracle.com>\newline

 } \\
\hline
fbb119dad56d & 19 May 2023 & Parav Pandit & { transport-pci: Avoid first vq index reference
\newline


Drop reference to first virtqueue as it is already
covered now by the generic section in first patch.
\newline

Fixes: \url{https://github.com/oasis-tcs/virtio-spec/issues/163}\newline
Reviewed-by: David Edmondson <david.edmondson@oracle.com>\newline
Acked-by: Halil Pasic <pasic@linux.ibm.com>\newline
Signed-off-by: Parav Pandit <parav@nvidia.com>\newline
Signed-off-by: Michael S. Tsirkin <mst@redhat.com>\newline

 } \\
\hline
a7a21e451987 & 19 May 2023 & Parav Pandit & { transport-mmio: Rename QueueNum register
\newline


These are further named differently between pci and mmio transport.
PCI transport indicates queue size as queue_size.
\newline

To bring consistency between pci and mmio transport,
rename the QueueNumMax and QueueNum
registers to QueueSizeMax and QueueSize respectively.
\newline

Fixes: \url{https://github.com/oasis-tcs/virtio-spec/issues/163}\newline
Reviewed-by: Cornelia Huck <cohuck@redhat.com>\newline
Reviewed-by: Jiri Pirko <jiri@nvidia.com>\newline
Reviewed-by: Halil Pasic <pasic@linux.ibm.com>\newline
Signed-off-by: Parav Pandit <parav@nvidia.com>\newline
Signed-off-by: Michael S. Tsirkin <mst@redhat.com>\newline

 } \\
\hline
9ddc59553984 & 19 May 2023 & Parav Pandit & { transport-mmio: Avoid referring to zero based index
\newline


VQ range is already described in the first patch in basic virtqueue
section. Hence remove the duplicate reference to it.
\newline

Fixes: \url{https://github.com/oasis-tcs/virtio-spec/issues/163}\newline
Reviewed-by: David Edmondson <david.edmondson@oracle.com>\newline
Acked-by: Halil Pasic <pasic@linux.ibm.com>\newline
Signed-off-by: Parav Pandit <parav@nvidia.com>\newline
Signed-off-by: Michael S. Tsirkin <mst@redhat.com>\newline

 } \\
\hline
e7a764f66598 & 19 May 2023 & Parav Pandit & { transport-ccw: Rename queue depth/size to other transports
\newline


max_num field reflects the maximum queue size/depth. Hence align name of
this field with similar field in PCI and MMIO transport to
max_queue_size.
Similarly rename 'num' to 'size'.
\newline

Fixes: \url{https://github.com/oasis-tcs/virtio-spec/issues/163}\newline
Reviewed-by: Halil Pasic <pasic@linux.ibm.com>\newline
Signed-off-by: Parav Pandit <parav@nvidia.com>\newline
Signed-off-by: Michael S. Tsirkin <mst@redhat.com>\newline

 } \\
\hline
c3092410ac51 & 19 May 2023 & Parav Pandit & { transport-ccw: Refer to the vq by its index
\newline


Currently specification uses virtqueue index and
number interchangeably to refer to the virtqueue.
\newline

Instead refer to it by its index.
\newline

Fixes: \url{https://github.com/oasis-tcs/virtio-spec/issues/163}\newline
Reviewed-by: Halil Pasic <pasic@linux.ibm.com>\newline
Signed-off-by: Parav Pandit <parav@nvidia.com>\newline
Signed-off-by: Michael S. Tsirkin <mst@redhat.com>\newline

 } \\
\hline
d6f310dbb3bf & 19 May 2023 & Parav Pandit & { virtio-net: Avoid duplicate receive queue example
\newline


Receive queue number/index example is duplicate which is already defined
in the Setting RSS parameters section.
\newline

Hence, avoid such duplicate example and prepare it for the subsequent
patch to describe using receive queue handle.
\newline

Fixes: \url{https://github.com/oasis-tcs/virtio-spec/issues/163}\newline
Reviewed-by: Cornelia Huck <cohuck@redhat.com>\newline
Signed-off-by: Parav Pandit <parav@nvidia.com>\newline
Signed-off-by: Michael S. Tsirkin <mst@redhat.com>\newline

 } \\
\hline
da0e16928d0b & 19 May 2023 & Parav Pandit & { virtio-net: Describe RSS using rss rq id
\newline


The content of the indirection table and unclassified_queue were
originally described based on mathematical operations. In order to
make it easier to understand and to avoid intermixing the array
index with the vq index, introduce a structure
rss_rq_id (RSS receive queue
ID) and use it to describe the unclassified_queue and
indirection_table fields.
\newline

As part of it, have the example that uses non-zero virtqueue
index which helps to have better mapping between receiveX
object with virtqueue index and the actual value in the
indirection table.
\newline

Fixes: \url{https://github.com/oasis-tcs/virtio-spec/issues/163}\newline
Reviewed-by: David Edmondson <david.edmondson@oracle.com>\newline
Signed-off-by: Parav Pandit <parav@nvidia.com>\newline
Signed-off-by: Michael S. Tsirkin <mst@redhat.com>\newline

 } \\
\hline
f9ff777fba59 & 19 May 2023 & Parav Pandit & { virtio-net: Update vqn to vq_index for cvq cmds
\newline


Replace field name vqn to vq_index for recent virtqueue level commands.
\newline

Fixes: \url{https://github.com/oasis-tcs/virtio-spec/issues/163}\newline
Reviewed-by: David Edmondson <david.edmondson@oracle.com>\newline
Signed-off-by: Parav Pandit <parav@nvidia.com>\newline
Signed-off-by: Michael S. Tsirkin <mst@redhat.com>\newline

 } \\
\hline
74460ef69d5f & 19 May 2023 & Parav Pandit & { transport-mmio: Replace virtual queue with virtqueue
\newline


Basic facilities define the virtqueue construct for device <-> driver
communication.
\newline

PCI transport and individual devices description also refers to it as
virtqueue.
\newline

MMIO refers to it as 'virtual queue'.
\newline

Align MMIO transport description to call such object a virtqueue.
\newline

Fixes: \url{https://github.com/oasis-tcs/virtio-spec/issues/168}\newline
Reviewed-by: Stefan Hajnoczi <stefanha@redhat.com>\newline
Signed-off-by: Parav Pandit <parav@nvidia.com>\newline
Signed-off-by: Michael S. Tsirkin <mst@redhat.com>\newline

 } \\
\hline
92295f3cb947 & 24 May 2023 & Michael S. Tsirkin & { fix: content: Rename confusing queue_notify_data and vqn names
\newline


when applying the patch, I omitted adding notifications-data-le.c
As a result, build fails.
\newline

Fixes: \url{https://github.com/oasis-tcs/virtio-spec/issues/172}\newline
Message-ID: <20230505014614.571520-4-parav@nvidia.com>\newline
Signed-off-by: Michael S. Tsirkin <mst@redhat.com>\newline

cc4a5604b259b3d6e18d50748423177b8eda3288
\newline

 } \\
\hline
6724756eaf0a & 07 Jul 2023 & Parav Pandit & { admin: Split opcode table rows with a line
\newline


Currently all opcode appears to be in a single row.
Separate them with a line similar to other tables.
\newline

Signed-off-by: Parav Pandit <parav@nvidia.com>\newline
Reviewed-by: Cornelia Huck <cohuck@redhat.com>\newline
[CH: pushed as editorial update]
Signed-off-by: Cornelia Huck <cohuck@redhat.com>\newline

 } \\
\hline
1518c9ce2cde & 07 Jul 2023 & Parav Pandit & { admin: Fix section numbering
\newline


Requirements are put one additional level down. Fix it.
\newline

Signed-off-by: Parav Pandit <parav@nvidia.com>\newline
Reviewed-by: Cornelia Huck <cohuck@redhat.com>\newline
[CH: pushed as editorial update]
Signed-off-by: Cornelia Huck <cohuck@redhat.com>\newline

 } \\
\hline
9c3ba1ec6acb & 14 Jul 2023 & Heng Qi & { virtio-net: support inner header hash
\newline


1. Currently, a received encapsulated packet has an outer and an inner header, but
the virtio device is unable to calculate the hash for the inner header. The same
flow can traverse through different tunnels, resulting in the encapsulated
packets being spread across multiple receive queues (refer to the figure below).
However, in certain scenarios, we may need to direct these encapsulated packets of
the same flow to a single receive queue. This facilitates the processing
of the flow by the same CPU to improve performance (warm caches, less locking, etc.).
\newline

               client1                    client2
                  |        +-------+         |
                  +------->|tunnels|<--------+
                           +-------+
                              |  |
                              v  v
                      +-----------------+
                      | monitoring host |
                      +-----------------+
\newline

To achieve this, the device can calculate a symmetric hash based on the inner headers
of the same flow.
\newline

2. For legacy systems, they may lack entropy fields which modern protocols have in
the outer header, resulting in multiple flows with the same outer header but
different inner headers being directed to the same receive queue. This results in
poor receive performance.
\newline

To address this limitation, inner header hash can be used to enable the device to advertise
the capability to calculate the hash for the inner packet, regaining better receive performance.
\newline

Fixes: \url{https://github.com/oasis-tcs/virtio-spec/issues/173}\newline

Signed-off-by: Heng Qi <hengqi@linux.alibaba.com>\newline
Reviewed-by: Xuan Zhuo <xuanzhuo@linux.alibaba.com>\newline
Reviewed-by: Parav Pandit <parav@nvidia.com>\newline
[CH: added missing lstlisting and hyperref escapes, fixed references]
Signed-off-by: Cornelia Huck <cohuck@redhat.com>\newline

 } \\
\hline
73c2fd96af96 & 17 Jul 2023 & Haixu Cui & { virtio-spi: define the DEVICE ID for virtio SPI master
\newline


Define the DEVICE ID of virtio SPI master device as 45.
\newline

Fixes: \url{https://github.com/oasis-tcs/virtio-spec/issues/174}\newline
Signed-off-by: Cornelia Huck <cohuck@redhat.com>\newline

 } \\
\hline
03c2d32e5093 & 21 Jul 2023 & Parav Pandit & { admin: Add group member legacy register access commands
\newline


Introduce group member legacy common configuration and legacy device
configuration access read/write commands.
\newline

Group member legacy registers access commands enable group owner driver
software to access legacy registers on behalf of the guest virtual
machine.
\newline

Usecase:\newline
========
1. A hypervisor/system needs to provide transitional
   virtio devices to the guest VM at scale of thousands,
   typically, one to eight devices per VM.
\newline

2. A hypervisor/system needs to provide such devices using a
   vendor agnostic driver in the hypervisor system.
\newline

3. A hypervisor system prefers to have single stack regardless of
   virtio device type (net/blk) and be future compatible with a
   single vfio stack using SR-IOV or other scalable device
   virtualization technology to map PCI devices to the guest VM.
   (as transitional or otherwise)
\newline

Motivation/Background:
=====================
The existing virtio transitional PCI device is missing support for
PCI SR-IOV based devices. Currently it does not work beyond
PCI PF, or as software emulated device in reality. Currently it
has below cited system level limitations:
\newline

[a] PCIe spec citation:
VFs do not support I/O Space and thus VF BARs shall not indicate I/O Space.
\newline

[b] cpu arch citiation:
Intel 64 and IA-32 Architectures Software Developer’s Manual:
The processor’s I/O address space is separate and distinct from
the physical-memory address space. The I/O address space consists
of 64K individually addressable 8-bit I/O ports, numbered 0 through FFFFH.
\newline

[c] PCIe spec citation:
If a bridge implements an I/O address range,...I/O address range will be
aligned to a 4 KB boundary.
\newline

Overview:\newline
=========
Above usecase requirements is solved by PCI PF group owner accessing
its group member PCI VFs legacy registers using the administration
commands of the group owner PCI PF.
\newline

Two types of administration commands are added which read/write PCI VF
registers.
\newline

Software usage example:
=======================
\newline

1. One way to use and map to the guest VM is by using vfio driver
framework in Linux kernel.
\newline

                +----------------------+
                |pci_dev_id = 0x100X   |
+---------------|pci_rev_id = 0x0      |-----+
|vfio device    |BAR0 = I/O region     |     |
|               |Other attributes      |     |
|               +----------------------+     |
|                                            |
+   +--------------+     +-----------------+ |
|   |I/O BAR to AQ |     | Other vfio      | |
|   |rd/wr mapper\& |     | functionalities | |
|   | forwarder    |     |                 | |
|   +--------------+     +-----------------+ |
|                                            |
+------+-------------------------+-----------+
       |                         |
   Config region                 |
     access                Driver notifications
       |                         |
  +----+------------+       +----+------------+
  | +-----+         |       | PCI VF device A |
  | | AQ  |-------------+---->+-------------+ |
  | +-----+         |   |   | | legacy regs | |
  | PCI PF device   |   |   | +-------------+ |
  +-----------------+   |   +-----------------+
                        |
                        |   +----+------------+
                        |   | PCI VF device N |
                        +---->+-------------+ |
                            | | legacy regs | |
                            | +-------------+ |
                            +-----------------+
\newline

2. Continue to use the virtio pci driver to bind to the
   listed device id and use it as in the host.
\newline

3. Use it in a light weight hypervisor to run bare-metal OS.
\newline

Fixes: \url{https://github.com/oasis-tcs/virtio-spec/issues/167}\newline
Signed-off-by: Parav Pandit <parav@nvidia.com>\newline
Signed-off-by: Michael S. Tsirkin <mst@redhat.com>\newline
Signed-off-by: Cornelia Huck <cohuck@redhat.com>\newline

 } \\
\hline
7fe9191b9666 & 04 Aug 2023 & Cornelia Huck & { edit: remove old changelog
\newline


Move it to cl-cs02-12.tex.
\newline

Signed-off-by: Cornelia Huck <cohuck@redhat.com>\newline

 } \\
\hline
f023dfa76316 & 09 Aug 2023 & Cornelia Huck & { edit: add changelog for 1.3
\newline


Signed-off-by: Cornelia Huck <cohuck@redhat.com>\newline

 } \\
\hline
2985fdc7676f & 25 Aug 2023 & Michael S. Tsirkin & { changelog: tweak column width
\newline


make description column wide and the others narrow
likely won't work well if we change page size but oh well
\newline

[CH: tweaked the tweak]
Signed-off-by: Michael S. Tsirkin <mst@redhat.com>\newline
Signed-off-by: Cornelia Huck <cohuck@redhat.com>\newline

 } \\
\hline
6234e8719405 & 25 Aug 2023 & Michael S. Tsirkin & { changelog: formatting fixes
\newline


Used lstlisting for ascii art. Note: if we want to we can also
break up large ascii art in the last half.
Forced no index on patch subject (looked ugly to me).
Added vspace after subject and before signature tags.
Reformatted nested lists in one commit using itemize (did not bother with
all of them but be my guest).
Liberally added paragraph breaks where it seemed appropriate.
\newline

[CH: added a bunch more enumerate/itemize, except where I was not able
to figure it out]
Signed-off-by: Michael S. Tsirkin <mst@redhat.com>\newline
Signed-off-by: Cornelia Huck <cohuck@redhat.com>\newline

 } \\
\hline
f23778dea9ed & 25 Aug 2023 & Michael S. Tsirkin & { work around extra row
\newline


latex seems to think we have an extra row in the table.
could not figure out why, but at least let's make it look
cleaner by adding cell boundaries.
\newline

Signed-off-by: Michael S. Tsirkin <mst@redhat.com>\newline
Signed-off-by: Cornelia Huck <cohuck@redhat.com>\newline

 } \\
\hline
c28a45d100bc & 25 Aug 2023 & Cornelia Huck & { acknowledgements: update for 1.3
\newline


Move some names to the section for previous versions, add names of new
contributors, etc.
\newline

Signed-off-by: Cornelia Huck <cohuck@redhat.com>\newline

 } \\
\hline
8e661c999085 & 25 Aug 2023 & Cornelia Huck & { editorial: update copyright date in PDF footer to 2023
\newline


Signed-off-by: Cornelia Huck <cohuck@redhat.com>\newline

 } \\
\hline
d401d40faf81 & 25 Aug 2023 & Cornelia Huck & { change revisions: diff from v1.2, current v1.3
\newline


Signed-off-by: Cornelia Huck <cohuck@redhat.com>\newline

 } \\
\hline
0f8579cec4c5 & 25 Aug 2023 & Cornelia Huck & { title: note that 1.3 supercedes 1.2
\newline


Signed-off-by: Cornelia Huck <cohuck@redhat.com>\newline

 } \\
\hline
1547af778060 & 25 Aug 2023 & Cornelia Huck & { revision: update date
\newline


Signed-off-by: Cornelia Huck <cohuck@redhat.com>\newline

 } \\
\hline
